\documentclass[../full]{subfiles}


\begin{document}
    \MainEx[Homotopie von Kurven]{5}

    Wir suchen Kurven von~\( \IntervalClosed{0}{2} \) nach~\( \RealNums^2 \),
    gegeben durch die Punkte
    \begin{equation*}
        p_0 = \begin{pmatrix} 1 \\ 0 \end{pmatrix}
        , \quad
        p_1 = \begin{pmatrix} 2 \\ 3 \end{pmatrix}
        , \quad
        p_2 = \begin{pmatrix} -1 \\ -1 \end{pmatrix}
    \end{equation*}


    \PartEx[Polygonzug]{2}

    Ein Polygonzug durch diese Punkte ist gegeben durch
    \begin{equation*}
        \gamma_1 \colon \IntervalClosed{0}{2} \to \RealNums^2
        , \enspace
        t \mapsto \begin{cases}
            \paren{1 - t} p_0 + t p_1 & t \in \IntervalClosed{0}{1} \\
            \paren{2 - t} p_1 + \paren{t - 1} p_2 & t \in \IntervalOC{1}{2}
        \end{cases}
    \end{equation*}


    \PartEx[Stetig Differenzierbare Kurve]{2}

    Eine beliebig oft differenzierbare Kurve durch diese Punkte
    erhalten wir beispielsweise
    indem wir die Komponenten durch Polynome beschreiben.

    Wir betrachten ein Polynom~\( a x^2 + bx + c = y_x \)
    zu den gew\"unschten Werten
    und erhalten auf diese Weise ein Lineares Gleichungssystem:
    \begin{equation*}
        \left. \begin{array}{ @{} r @{{}+{}} r @{{}+{}} r @{{}={}} l @{} }
            \multicolumn{2}{c}{} & c & y_0 \\
             a &  b & c & y_1 \\
            4a & 2b & c & y_2
        \end{array} \right\rbrace
        \left. \begin{array}{ @{} r @{{}+{}} r @{{}={}} l @{} }
             a &  b & y_1 - y_0 \\
            4a & 2b & y_2 - y_0
        \end{array} \right\rbrace
        \begin{array}{ @{} r @{{}={}} l @{} }
            2a & y_2 + y_0 - 2 y_1 \\
            2b & 4 y_1 - y_2 - 3 y_0
        \end{array}
    \end{equation*}
    So erhalten wir eine unendlich oft differenzierbare Kurve
    \begin{equation*}
        \gamma_2 \colon \IntervalClosed{0}{2} \to \RealNums^2
        , \enspace
        t \mapsto \begin{pmatrix}
            -2 t^2 + 3t + 1 \\
            -\frac{7}{2} t^2 + \frac{13}{2} t
        \end{pmatrix}
        , \qquad
        \gamma_2 \paren{i} = \gamma_1 \paren{i} = p_i, \enspace i = 0, 1, 2
    \end{equation*}



    \PartEx[Homotopie]{1}

    Wir suchen eine stetig verlaufende Homotopie~\(
        \Phi \colon
            \IntervalClosed{0}{2} \times \IntervalClosed{0}{1} \to \RealNums^2
    \)
    mit~\( \Phi \paren{t, 0} = \gamma_1 \paren{t} \)
    und~\( \Phi \paren{t, 1} = \gamma_2 \paren{t} \).
    Eine solche Homotopie ist
    \begin{equation*}
        \Phi \colon
            \IntervalClosed{0}{2} \times \IntervalClosed{0}{1} \to \RealNums^2
        , \enspace
        \begin{pmatrix} t \\ s \end{pmatrix}
        \mapsto \paren{1 - s} \gamma_1 \paren{t} + s \gamma_2 \paren{t}
    \end{equation*}
\end{document}
