\documentclass[../full]{subfiles}


\begin{document}
    \MainEx[Volumina von Parallelotopen]{2}

    Zu gegebenen Vektoren aus~\( \RealNums^4 \)
    \begin{equation*}
        x_1 = \begin{pmatrix} 1 \\ 0 \\ 1 \\ 0 \end{pmatrix}
        , \quad
        x_2 = \begin{pmatrix} 1 \\ 1 \\ 0 \\ 1 \end{pmatrix}
        , \quad
        x_3 = \begin{pmatrix}
            \phantom{-}1 \\ -1 \\ -1 \\ \phantom{-}0
        \end{pmatrix}
    \end{equation*}
    suchen wir die Volumina von aufgespannten Parallelotopen.

    \begin{align*}
        \volume[\big]{P \paren{x_1, x_2}} &
        = \sqrt{ \det \paren[\Bigger]{
            \begin{pmatrix} 1 & 0 & 1 & 0 \\ 1 & 1 & 0 & 1 \end{pmatrix}
            \begin{pmatrix} 1 & 1 \\ 0 & 1 \\ 1 & 0 \\ 0 & 1 \end{pmatrix}
        } }
        = \sqrt{ \begin{vmatrix} 2 & 1 \\ 1 & 3 \end{vmatrix} }
        = \sqrt{5}
        \\
        \volume[\big]{P \paren{x_1, x_2, x_3}} &
        = \sqrt{ \det \paren[\Bigger]{
            \begin{pmatrix}
                1 & \phantom{-}0 & \phantom{-}1 & 0 \\
                1 & \phantom{-}1 & \phantom{-}0 & 1 \\
                1 & -1 & -1 & 0
            \end{pmatrix}
            \begin{pmatrix}
                1 & 1 & \phantom{-}1 \\
                0 & 1 & -1 \\
                1 & 0 & -1 \\
                0 & 1 & \phantom{-}0
            \end{pmatrix}
        } }
        = \sqrt{
            \begin{vmatrix} 2 & 1 & 0 \\ 1 & 3 & 0 \\ 0 & 0 & 3 \end{vmatrix}
        }
        = \sqrt{15}
        = \sqrt{3} \sqrt{5}
    \end{align*}
\end{document}
