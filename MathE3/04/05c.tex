\documentclass[../full]{subfiles}


\begin{document}
    \MainEx[Koordinatentransformation und Dichte]{6}
    \label{task:coordinate_transform:cylinder}

    Zu~\( H > 0 \) sei ein K\"orper
    mit Dichtefunktion~\( \rho \paren{x, y, z} \)
    beschrieben durch
    \begin{equation*}
        K = \set[\Bigger]{
            \begin{pmatrix} x \\ y \\ z \end{pmatrix} \in \RealNums^3
            \given
                \begin{matrix}
                    1 \leq x^2 + y^2 \leq 2, \\ 0 \leq z \leq H
                \end{matrix}
        }
    \end{equation*}


    \PartEx[Skizze]{1}
    \label{task:coordinate_transform:cylinder:sketch}

    \begin{figure}
        \centering
        \HomeworkPart[Sketch]{Pipe}

        \caption*{
            Skizze zu
            \hyperref[task:coordinate_transform:cylinder:sketch]{%
                \ref*{task:coordinate_transform:cylinder}%
                ~\ref*{task:coordinate_transform:cylinder:sketch}%
            }
        }
        \label{task:coordinate_transform:cylinder:sketch:visual}
    \end{figure}

    \( K \)~beschreibt einen Hohlzylinder
    entlang der \( z \)\nobreak\mbox{-}\nobreak\hskip0pt\relax Achse
    mit H\"ohe~\( H \), Innenradius~\( 1 \), Au\ss enradius~\( \sqrt{2} \).
    Eine entsprechende Skizze findet sich auf
    \hyperref[task:coordinate_transform:cylinder:sketch:visual]{%
        Seite~\pageref*{task:coordinate_transform:cylinder:sketch:visual}%
    }.


    \PartEx[Koordinatentransformation]{2}

    Durch die Struktur von~\( K \)
    bietet sich die Transformation in Zylinerkoordinaten an.
    \begin{gather*}
        \Phi \colon
            \RealNumsZeroPos \times \IntervalCO{0}{2 \pi} \times \RealNums
            \to \RealNums^3,
        \enspace
        \begin{pmatrix} r \\ \varphi \\ z \end{pmatrix}
        \mapsto \begin{pmatrix}
            r \cos \varphi \\ r \sin \varphi \\ z
        \end{pmatrix}
        \displaybreak[0] \\
        \TotalD{\Phi}{r, \varphi, z}
        = \begin{pmatrix}
            \cos \varphi & -r \sin \varphi           & 0 \\
            \sin \varphi & \phantom{-}r \cos \varphi & 0 \\
            0            & 0                         & 1
        \end{pmatrix}
        , \qquad
        \det \paren[\big]{\TotalD{\Phi}{r, \varphi, z}}
        = r \cos^2 \paren{\varphi} + r \sin^2 \paren{\varphi}
        = r
    \end{gather*}
    Wir sehen dass f\"ur die Transformation~\( \Phi \)
    eine Menge~\( \widetilde{K} \) existiert
    so dass~\( \Phi \paren{\widetilde{K}} = K \).
    \begin{equation*}
        \widetilde{K} = \IntervalClosed[\big]{1}{\sqrt{2}}
            \times \IntervalCO[\big]{0}{2 \pi}
            \times \IntervalClosed[\big]{0}{H}
    \end{equation*}

    Damit k\"onnen wir zu gegebener Dichtefunktion~\( \rho \)
    die Masse~\( M_K \) beschreiben.
    Hierbei sei~\( \widetilde{\rho} \coloneqq \rho \circ \Phi \).
    \begin{equation*}
        M_K = \int_K \rho \paren{x} \dx
        = \int_{\widetilde{K}}
            \widetilde{\rho} \paren{\widetilde{x}}
            \abs[\Big]{\det \paren[\big]{\TotalD{\Phi}{\widetilde{x}}}}
        \dx[\widetilde{x}]
        = \int_{z = 0}^H
            \int_{r = 1}^{\sqrt{2}}
                r
                \int_{\varphi = 0}^{2 \pi}
                    \rho \paren[\big]{
                        r \cos \paren{\varphi}, r \sin \paren{\varphi}, z
                    }
                \dx[\varphi]
            \dx[r]
        \dx[z]
    \end{equation*}


    \PartEx[Produktform der Dichtefunktion]{1}

    Wenn die Dichtefunktion~\(
        \rho \paren{x, y, z} = \rho_1 \paren{z} \rho_2^* \paren{x, y}
    \) dann l\"asst sich dieses Integral weiter vereinfachen:
    \begin{equation*}
        M_K
        = \int_0^H \rho_1 \paren{z} \dx[z]
        \int_{r = 1}^{\sqrt{2}}
            r
            \int_{\varphi = 0}^{2 \pi}
                \rho_2^* \paren{
                    r \cos \paren{\varphi}, r \sin \paren{\varphi}
                }
            \dx[\varphi]
        \dx[r]
    \end{equation*}
    Wenn weiterhin gilt dass~\(
        \rho_2^* \paren{x, y} = \rho_2 \paren[\big]{\sqrt{x^2 + y^2}}
    \), dann erhalten wir
    \begin{equation*}
        M_K = 2 \pi \int_0^H \rho_1 \paren{z} \dx[z]
            \int_1^{\sqrt{2}} r \rho \paren{r} \dx[r]
    \end{equation*}


    \PartEx[Masse und Schwerpunkt]{2}

    Wenn die Dichtefunktion gegeben ist durch~\(
        \rho \paren{x, y, z} = z \paren{x^2 + y^2}
    \),
    dann ist~\( \rho_1 \paren{x} = x \) und~\( \rho_2 \paren{x} = x^2 \).
    Damit ergeben sich Masse und Schwerpunkt zu
    \begin{align*}
        M_K &= 2 \pi \int_0^H z \dx[z] \int_1^{\sqrt{2}} r^3 \dx[r]
        = 2 \pi
            \FunctionBounds{\frac{z^2}{2}}{0}{H}
            \FunctionBounds{\frac{r^4}{4}}{1}{\sqrt{2}}
        = \paren[\Big]{1 - \frac{1}{4}} H^2 \pi
        = \frac{3}{4} H^2 \pi
        \displaybreak[0] \\
        S_K &= \frac{1}{M_K} \int_K x \rho \paren{x} \dx
        = \frac{4}{3 H^2 \pi} \int_{\widetilde{K}}
            \Phi \paren{\widetilde{x}}
            \rho \paren[\big]{\Phi \paren{\widetilde{x}}}
            \abs[\Big]{\det \paren[\big]{\TotalD{\Phi}{\widetilde{x}}}}
        \dx[\widetilde{x}]
        = \frac{4}{3 H^2 \pi} \int_{\widetilde{K}}
            \begin{psmallmatrix}
                \paren{z r^3} r \cos \varphi \\
                \paren{z r^3} r \sin \varphi \\
                \paren{z r^3} z
            \end{psmallmatrix}
        \dx[\paren{r, \varphi, z}]
        \\ &
        = \frac{4}{3 H^2 \pi} \int_0^H \int_1^{\sqrt{2}} \int_0^{2 \pi}
            \begin{psmallmatrix}
                z r^4 \cos \varphi \\
                z r^4 \sin \varphi \\
                z^2 r^3
            \end{psmallmatrix}
        \dx[\varphi] \dx[r] \dx[z]
        = \frac{4}{3 H^2 \pi} \int_0^H \int_1^{\sqrt{2}}
            \begin{psmallmatrix}
                0 \\ 0 \\ 2 z^2 r^3 \pi
            \end{psmallmatrix}
        \dx[r] \dx[z]
        \\ &
        = \frac{8}{3 H^2} \StandardVect{3}
            \int_0^H z^2 \dx[z]
            \int_1^{\sqrt{2}} r^3 \dx[r]
        = \frac{8}{3 H^2} \StandardVect{3}
            \cdot \FunctionBounds{\frac{z^3}{3}}{0}{H}
            \cdot \FunctionBounds{\frac{r^4}{4}}{1}{\sqrt{2}}
        = \frac{8}{3 H^2} \StandardVect{3} \cdot \frac{H^3}{3} \cdot \frac{3}{4}
        = \frac{2}{3} H \StandardVect{3}
    \end{align*}
\end{document}
