\documentclass[../full]{subfiles}

\newcommand\ValuePoint[1]{\mathtt{#1}}
\NewDocumentCommand\ValuePoints{ m g }{
    \ValuePoint{#1}
    \IfValueT{#2}{, \ValuePoints{#2}}
}


\begin{document}
    \MainEx[Parametrisierung von Kurven]{3}

    Gegeben sind die Punkte~\( \ValuePoints{P}{Q}{R} \in \RealNums^2 \),
    die in dieser Reihenfolge durchlaufen werden sollen.
    Hierbei sind die Werte der Punkte
    \begin{equation*}
        \ValuePoint{P} = \begin{pmatrix} 0 \\ 0 \end{pmatrix}
        , \qquad
        \ValuePoint{Q} = \begin{pmatrix} 1 \\ 0 \end{pmatrix}
        , \qquad
        \ValuePoint{R} = \begin{pmatrix} 0 \\ 1 \end{pmatrix}
    \end{equation*}


    \PartEx[Polygonzug]{1}

    Zu finden sei eine stetige Kurve~\(
        \gamma \colon \IntervalClosed{a}{b} \to \RealNums^2
    \), die diese Punkte durchl\"auft.

    Hierbei ist die einfachste L\"osung ein Polygonzug,
    skaliert auf das Intervall~\( \IntervalClosed{a}{b} \).
    \begin{gather*}
        \begin{aligned}
            P_{\ValuePoints{P}{Q}{R}} \colon
                \IntervalClosed{0}{2} \to \RealNums^2
                , \enspace &
                t \mapsto \begin{dcases}
                    \paren{1 - t} \ValuePoint{P} + t \, \ValuePoint{Q}
                        & t \in \IntervalClosed{0}{1}
                    \\
                    \paren{2 - t} \ValuePoint{Q} + \paren{t - 1} \ValuePoint{R}
                        & t \in \IntervalClosed{1}{2}
                \end{dcases}
            \\
            h_{a, b} \coloneqq \widetilde{h} \colon
                \IntervalClosed{a}{b} \to \IntervalClosed{0}{2}
                , \enspace &
                t \mapsto \frac{t - a}{b - a} \cdot 2
                , \qquad
                \widetilde{h}^{-1} \paren{t} = \frac{t \paren{b - a}}{2} + a
        \end{aligned}
        \displaybreak[0] \\
        \gamma = P_{\ValuePoints{P}{Q}{R}} \circ \widetilde{h}
    \end{gather*}


    \PartEx[L\"ange der Kurve]{1}

    Weil die Skalierung des Polygonzugs
    eine simple Form der Umparametrisierung darstellt,
    gilt
    \begin{equation*}
        L \paren{\gamma}
        = L \paren{P_{\ValuePoints{P}{Q}{R}}}
        = L \paren{P_{\ValuePoints{P}{Q}} + P_{\ValuePoints{Q}{R}}}
        = L \paren{P_{\ValuePoints{P}{Q}}} + L \paren{P_{\ValuePoints{Q}{R}}}
        = \norm[\big]{\ValuePoint{Q} - \ValuePoint{P}}
            + \norm[\big]{\ValuePoint{R} - \ValuePoint{Q}}
        = 1 + \sqrt{2}
    \end{equation*}


    \PartEx[Parametrisierung nach der Bogenl\"ange]{1}

    Wir suchen eine Umparametrisierung~\(
        \widehat{\gamma} \colon
            \IntervalClosed[\big]{0}{L \paren{\gamma}} \to \RealNums^2
    \)
    so dass
    mit Ausnahme von einer endlichen Zahl an Undefiniertheitsstellen
    gilt dass~\( \norm[\big]{\widehat{\gamma}' \paren{t}} = 1 \)\footnote{
        Das Aufgabenblatt verwendet hier~\( \norm{\gamma' \paren{t}} \),
        was wohl ein Druckfehler ist.
    }.
    Hierzu parametrisieren wir der Einfachheit halber den Polygonzug um,
    den wir bereits verwendet haben um~\( \gamma \) zu erzeugen.

    Wir betrachten die Bogenl\"ange
    der zwei zusammensetzenden Polygonz\"uge
    \begin{gather*}
        \begin{aligned}
            S_1 \paren{t} & \coloneqq
            \int_0^t \norm[\big]{P_{\ValuePoints{P}{Q}} \paren{t}} \dx[t]
            = \int_0^t 1 \dx[t] = t,
                & \quad & \quad
                & S_1^{-1} \paren{t} &= t
            \\
            S_2 \paren{t} & \coloneqq
            \int_0^t \norm[\big]{P_{\ValuePoints{Q}{R}} \paren{t}} \dx[t]
            = \int_0^t \sqrt{2} \dx[t] = t \sqrt{2},
                & \quad & \quad
                & S_2^{-1} \paren{t} &= \frac{t}{\sqrt{2}}
        \end{aligned}
        \displaybreak[0] \\
        \widehat{h} \paren{t} \colon
            \IntervalClosed[\big]{0}{1 + \sqrt{2}} \to \IntervalClosed{0}{2}
            , \enspace
            t \mapsto \begin{cases}
                S_1^{-1} \paren{t} = t
                    & t \in \IntervalClosed{0}{1}
                \\
                1 + S_2^{-1} \paren{t - 1} = 1 + \frac{t - 1}{\sqrt{2}}
                    & t \in \IntervalClosed{1}{1 + \sqrt{2}}
            \end{cases}
        \displaybreak[0] \\
        \widehat{\gamma}
        = P_{\ValuePoints{P}{Q}{R}} \circ \widehat{h}
        = P_{\ValuePoints{P}{Q}{R}} \circ \widetilde{h}
            \circ \widetilde{h}^{-1} \circ \widehat{h}
        = \gamma \circ \widetilde{h}^{-1} \circ \widehat{h}
    \end{gather*}


    \PartExNoScore[Stetig Differenzierbare Kurve]

    Wir suchen weiterhin eine stetig differenzierbare Kurve~\(
        \eta \colon \IntervalClosed{c}{d} \to \RealNums^2
    \) die ebenfalls diese Punkte in der vorgegebenen Reihenfolge durchl\"auft.

    Zuerst suchen wir nach Funktionen,
    die auf einem beliebigen Intervall
    die gew\"unschten Komponenten der Koordinaten durlaufen.
    Wir finden beispielsweise
    \begin{gather*}
        \begin{aligned}
            x \colon \IntervalClosed{0}{2} \to \RealNums, \enspace &
            t \mapsto -t \paren{t - 2} \\
            y \colon \IntervalClosed{0}{2} \to \RealNums, \enspace &
            t \mapsto \frac{1}{2} t \paren{t - 1}
        \end{aligned}
        \displaybreak[1] \\
        \widetilde{\eta} \colon
            \IntervalClosed{0}{2} \to \RealNums^2
            , \enspace
            t \mapsto \begin{pmatrix} x \paren{t} \\ y \paren{t} \end{pmatrix}
                = \begin{pmatrix}
                    -t \paren{t - 2} \\ \frac{1}{2} t \paren{t - 1}
                \end{pmatrix}
                = \begin{pmatrix}
                    -t^2 + 2t \\ \frac{1}{2} t^2 - \frac{1}{2} t
                \end{pmatrix}
        \displaybreak[0] \\
        \eta = \widetilde{\eta} \circ h_{c, d}
    \end{gather*}
    Weil~\( \eta \) eine Umparametrisierung von~\( \widetilde{\eta} \) ist,
    reicht es die L\"ange von~\( \widetilde{\eta} \) zu berechnen.
    \begin{equation*}
        \int_0^2 \norm[\big]{\widetilde{\eta}' \paren{t}} \dx[t]
        = \int_0^2 \sqrt{4 t^2 - 8t + 4 + t^2 - t + \frac{1}{4}} \dx[t]
        = \int_0^2 \sqrt{5 t^2 - 9t + \frac{17}{4}} \dx[t]
    \end{equation*}
    Den resultierenden Ausdruck zu integrieren ist zwar m\"oglich,
    aber relativ aufw\"andig.
    Ein Online-Integrator liefert als L\"osung
    \begin{align*}
        \int_0^2 \sqrt{5 t^2 - 9t + \frac{17}{4}} \dx[t] &
        = \FunctionBounds{
            \frac{
                4 \sqrt{5}
                \ln \paren[\big]{\sqrt{5} \sqrt{20 t^2 - 36t + 17} + 10t - 9}
                + \paren{50t - 45} \sqrt{20 t^2 - 36t + 17}
            }{200}
        }{0}{2}
        \\ &
        = \frac{1}{200} \paren[\Big]{
            4 \sqrt{5} \ln \paren[\big]{5 \sqrt{5} + 11} + 275
            - 4 \sqrt{5} \ln \paren[\big]{\sqrt{85} - 9} + 45 \sqrt{17}
        }
        \\ &
        = \frac{
            275 + 45 \sqrt{17} + 4 \sqrt{5} \paren[\big]{
                \ln \paren{5 \sqrt{5} + 11}
                - \ln \paren{\sqrt{85} - 9}
            }
        }{200}
        \approx 2.509
    \end{align*}
\end{document}
