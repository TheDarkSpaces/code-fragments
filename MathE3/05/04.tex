\documentclass[../full]{subfiles}

% https://tex.stackexchange.com/questions/9691/
% Avoid hyphenation in “2-D”
\newcommand\NoBreakHyphenate[2]{%
    #1%
    \nobreak\mbox{-}\nobreak\hskip0pt\relax%
    #2%
}


\begin{document}
    \MainEx[Arbeitsaufwand im Kraftfeld]{4}

    Wir wollen wissen,
    welche Arbeit aufgewendet werden muss
    um ein Teilchen durch ein Kraftfeld zu bewegen.
    Das Kraftfeld ist gegeben durch
    \begin{equation*}
        F \colon \RealNums^3 \to \RealNums^3
        , \enspace
        \begin{pmatrix} x \\ y \\ z \end{pmatrix}
            \mapsto
            \begin{pmatrix}
                3x - 4y - 2z \\ 4x + 2y - 3 z^2 \\ 2xz - 4 y^2 + z^3
            \end{pmatrix}
    \end{equation*}
    und das Teilchen wird entlang des Randes einer Ellipse bewegt.
    Die Ellipse liegt in der \NoBreakHyphenate{\( \paren{x, y} \)}{Ebene}
    um den Nullpunkt
    mit Halbachsen  \( 4 \)~in \NoBreakHyphenate{\( x \)}{Richtung}
    und \( 3 \)~in \NoBreakHyphenate{\( y \)}{Richtung}.
    Daraus ergibt sich f\"ur die Bewegung des Teilchens die Kurve
    \begin{equation*}
        \gamma_z \colon \IntervalClosed{\alpha}{\beta} \to \RealNums^3
            , \enspace
            t \mapsto \begin{pmatrix} 4 \cos t \\ 3 \sin t \\ z \end{pmatrix}
    \end{equation*}
    wobei~\( \alpha \) den Winkel zu Beginn der Bewegung
    und~\( \beta \) den Winkel am Ende der Bewegung beschreibt.
    Die Bewegung erfolgt entgegen dem Urzeigersinn,
    also gilt~\( \alpha < \beta \).

    Zur Berechnung der aufgewendeten Kraft
    verwenden wir das Kurvenintegral \ordinalnum{2}~Art
    \begin{align*}
        \int_{\gamma_z} F \vecds &
        = \int_\alpha^\beta
            \InnerProd[\Big]
                {F \paren[\big]{\gamma_z \paren{t}}}
                {\gamma_z' \paren{t}}
        \dx[t]
        = \int_\alpha^\beta
            \InnerProd[\Bigg]{
                \begin{pmatrix}
                    12 \cos \paren{t} - 12 \sin \paren{t} - 2z \\
                    16 \cos \paren{t} + 6 \sin \paren{t} - 3 z^2 \\
                    8z \cos \paren{t} - 36 \sin^2 \paren{t} + z^3
                \end{pmatrix}
            }{
                \begin{pmatrix} -4 \sin t \\ 3 \cos t \\ 0 \end{pmatrix}
            }
        \dx[t]
        \displaybreak[0] \\ &
        = \int_\alpha^\beta
            48
                - 30 \sin \paren{t} \cos \paren{t}
                + 8z \sin \paren{t}
                - 9 z^2 \cos \paren{t}
        \dx[t]
        \displaybreak[0] \\ &
        = 48 \paren{\beta - \alpha}
            - 30 \int_\alpha^\beta \sin \paren{t} \cos \paren{t} \dx[t]
            + 8z \int_\alpha^\beta \sin t \dx[t]
            - 9 z^2 \int_\alpha^\beta \cos t \dx[t]
        \\ &
        = 48 \paren{\beta - \alpha}
            - 30 \int_{\sin \alpha}^{\sin \beta} u \dx[u]
            - 8z \FunctionBounds{\cos t}{\alpha}{\beta}
            - 9 z^2 \FunctionBounds{\sin t}{\alpha}{\beta}
        \displaybreak[0] \\ &
        = 48 \paren{\beta - \alpha}
            - 30 \FunctionBounds{\frac{u^2}{2}}{\sin \alpha}{\sin \beta}
            - 8z \paren[\big]{\cos \paren{\beta} - \cos \paren{\alpha}}
            - 9 z^2 \paren[\big]{\sin \paren{\beta} - \sin \paren{\alpha}}
        \displaybreak[0] \\ &
        = 48 \paren{\beta - \alpha}
            - 15 \paren[\big]{\sin^2 \paren{\beta} - \sin^2 \paren{\alpha}}
            - 8z \paren[\big]{\cos \paren{\beta} - \cos \paren{\alpha}}
            - 9 z^2 \paren[\big]{\sin \paren{\beta} - \sin \paren{\alpha}}
    \end{align*}

    Wir sehen dass sich das Ergebnis
    bei \( n \in \NatNums \)~vollst\"andigen Umrundungen
    (also bei~\( \beta \coloneqq \alpha + 2n \pi \))
    vereinfacht zu~\( 96n \pi \).

    Weiterhin sehen wir,
    dass wir im Fall~\( z = 0 \) ein einfacheres Ergebnis erhalten
    f\"ur alle Werte von~\( \alpha, \beta \).
    \begin{equation*}
        \int_{\gamma_0} F \vecds
        = 48 \paren{\beta - \alpha}
            - 15 \paren[\big]{\sin^2 \paren{\beta} - \sin^2 \paren{\alpha}}
    \end{equation*}
\end{document}
