\documentclass[../full]{subfiles}


\begin{document}
    \MainEx[Integration und Potentialfelder]{3}

    Gegeben ist ein Vektorfeld
    \begin{equation*}
        F \colon \RealNums^3 \to \RealNums^3
            , \enspace
            \begin{pmatrix} x \\ y \\ z \end{pmatrix}
                \mapsto \begin{pmatrix}
                    1 + y^2 \\ 2xy + z^2 \\ 2yz
                \end{pmatrix}
    \end{equation*}


    \PartEx[Potential-Untersuchung]{2}

    Wenn~\( F \) ein Potentialfeld
    zu einem Potential~\( \varphi \colon \RealNums^3 \to \RealNums \) ist,
    dann muss~\(
        \TotalD{F}{x, y, z} = \HesseMatrix{\varphi} \paren{x, y, z}
    \) symmetrisch sein.
    Weiterhin muss \( \RealNums^3 \)~zusammenh\"angend sein,
    was aber ohnehin gegeben ist.
    Damit ist es hinreichend,
    \( \TotalD{F}{x, y, z} \)~auf Symmetrie zu untersuchen,
    um herauszufinden ob ein Potential~\( \varphi \) existiert.
    \begin{equation*}
        \TotalD{F}{x, y, z}
        = \begin{pmatrix}
            0  & 2y & 0  \\
            2y & 2x & 2z \\
            0  & 2z & 2y
        \end{pmatrix}
        \mathrlap{
            \enspace \checkmark \; \text{symmetrisch}
        }
    \end{equation*}

    Weil~\( F \) ein Potentialfeld ist, gilt
    \begin{equation*}
        F \paren{x, y, z}
        = \nabla \varphi \paren{x, y, z}
        = \begin{pmatrix}
            \frac{\partial \varphi}{\partial x} \paren{x, y, z} \\
            \frac{\partial \varphi}{\partial y} \paren{x, y, z} \\
            \frac{\partial \varphi}{\partial z} \paren{x, y, z}
        \end{pmatrix}
    \end{equation*}
    Daher reicht es aus,
    die einzelnen Komponenten zu integrieren
    und die Ergebnisse in ein Potential zusammenzuf\"uhren.
    \begin{gather*}
        \varphi \paren{x, y, z}
        = \int F_x \paren{x, y, z} \dx
        = \int F_y \paren{x, y, z} \dx[y]
        = \int F_z \paren{x, y, z} \dx[z]
        \displaybreak[0] \\
        \begin{aligned}
            \int F_x \paren{x, y, z} \dx &
            = \int 1 + y^2 \dx
            = x + x y^2 + c_x \paren{y, z}
            \\
            \int F_y \paren{x, y, z} \dx[y] &
            = \int 2xy + z^2 \dx[y]
            = x y^2 + y z^2 + c_y \paren{x, z}
            \\
            \int F_z \paren{x, y, z} \dx[z] &
            = \int 2yz \dx[z]
            = y z^2 + c_z \paren{x, y}
        \end{aligned}
        \displaybreak[0] \\
        \varphi \paren{x, y, z} = x + x y^2 + y z^2 \enspace + c \in \RealNums
    \end{gather*}


    \PartEx[Kurvenintegral im Potentialfeld]{1}

    Zu berechnen ist das Kurvenintegral \ordinalnum{2}~Art von~\( F \)
    \"uber eine Kurve
    \begin{equation*}
        \gamma \colon \IntervalClosed{0}{2} \to \RealNums^3
            , \enspace
            t \mapsto \begin{pmatrix} 1 \\ t \\ t^2 \end{pmatrix}
    \end{equation*}
    Weil~\( F \) das Potential~\( \varphi \) besitzt,
    k\"onnen wir die Berechnung vereinfachen zu
    \begin{equation*}
        \int_\gamma F \vecds
        = \varphi \paren[\big]{\gamma \paren{2}}
            - \varphi \paren[\big]{\gamma \paren{0}}
        = \varphi \paren{1, 2, 4} - \varphi \paren{1, 0, 0}
        = 1 + 4 + 32 - 1
        = 36
    \end{equation*}
\end{document}
