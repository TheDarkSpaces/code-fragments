\documentclass[../full]{subfiles}


\newcommand\Unit[2][\,]{
    \, \mathrm{#2}^{#1}
}
\newcommand\kg{\Unit{kg}}
\newcommand\m[1][]{\Unit[#1]{m}}
\newcommand\km[1][]{\Unit[#1]{km}}
\newcommand\Celsius{\Unit{\degree C}}
\newcommand\Kelvin{\Unit{K}}

\begin{document}
    \MainEx[Theobald's Ballonfahrt]{4}

    Der Drache Theobald m\"ochte mit einem Hei\ss luftballon verreisen,
    wozu er den Radius~\( R \) des kugelf\"ormigen Ballons ermitteln muss.
    Er wei\ss\ Folgendes:
    \begin{itemize}
        \item
        Er kennt das Volumen
        und die Oberfl\"ache einer Kugel mit Radius~\( R \).
        \begin{equation*}
            V \paren{R}
            \coloneqq \volume[\big]{S^2 \paren{R}} = \frac{4 R^3}{3} \pi
            , \qquad
            O \paren{R}
            \coloneqq \volume[\big]{\partial S^2 \paren{R}} = 4 R^2 \pi
        \end{equation*}
        Diese Werte sind die einzig relevanten f\"ur ihn.

        \item
        Das zu transportierende Gesamtgewicht betr\"agt
        in Abh\"angigkeit vom Radius des Ballons
        \begin{equation*}
            m \paren{R}
            = 2 200 \kg + \frac{3 \kg}{20 \m[2]} O \paren{R}
            = 2 200 \kg + \frac{3 \pi \kg R^2}{5 \m[2]}
        \end{equation*}

        \item
        Die Luftdichte auf der H\"ohe~\( h \) gen\"ugt der Formel
        \begin{equation*}
            \rho \paren{h}
            = \frac{13 \kg}{10 \m[3]} \cdot 2^{-\frac{2h}{11 \km}}
        \end{equation*}

        \item
        Theobald m\"ochte eine H\"ohe von wenigstens~\( 4 \km \) erreichen.

        Weiterhin reicht es ihm aus,
        die vertikale Ausdehnung zu vernachl\"assigen,
        also so zu rechnen
        als bef\"ande sich der Ballon auf einer einzigen H\"ohe.

        \item
        Die Temperatur-Differenz die Theobald
        zwischen der Luft innerhalb und au\ss erhalb des Ballons erreichen kann
        ist abh\"angig von der Oberfl\"ache des Ballons, also vom Radius.
        \begin{equation*}
            \Delta T \paren{R} = \frac{5 \cdot 10^4 \m[2]}{R^2} \Celsius
        \end{equation*}

        \item
        Die Au\ss entemperatur h\"angt ab von der H\"ohe,
        auf der sich Theobald befindet.
        \begin{equation*}
            T \paren{h} = 15 \Celsius - \frac{13h \Celsius}{2 \km}
        \end{equation*}

        \item
        Die Luftdichte im Inneren des Ballons~\( \rho_i \)
        ist abh\"angig von der Dichte der Au\ss enluft~\( \rho \),
        der Au\ss entemperatur~\( T_K \) in Kelvin,
        und der Temperaturdifferenz zwischen innen und au\ss en~\( \Delta T \).
        \begin{equation*}
            \rho_i \paren{\rho, T_K, \Delta T}
            = \frac{\rho}{1 + \frac{\Delta T}{T_K}}
        \end{equation*}

        Wir halten fest, dass f\"ur die Umrechnung der Temperatur gilt
        \begin{equation*}
            a \Celsius = \paren[\big]{a + \frac{5463}{20}} \Kelvin
        \end{equation*}
        Weil wir in~\( \Delta T \) nur einen Temperatur-Unterschied bestimmen
        und die Umrechnung von Grad Celsius nach Kelvin
        mit einem Faktor~\( 1 \) linear verl\"auft,
        \"andert sich mit der Umrechnung in Kelvin nur die Einheit.
        Damit erhalten wir nach Einsetzen der Parameter
        \begin{equation*}
            \rho_i \paren{R, h}
            = \frac
                {2^{-\frac{2h}{11 \km}} \frac{13 \kg}{10 \m[3]}}
                {
                    1 + \frac
                        {\frac{5 \cdot 10^4 \m[2]}{R^2} \Kelvin}
                        {
                            \paren{
                                \frac{30 \km - 13h}{2 \km} + \frac{5463}{20}
                            } \Kelvin
                        }
                }
            = 2^{-\frac{2h}{11 \km}} \frac{13 \kg}{10 \m[3]}
                \cdot \frac
                    {1}
                    {
                        1 + \frac
                            {1 \km \cdot 10^6 \m[2]}
                            {R^2 \paren{5763 \km - 130h}}
                    }
        \end{equation*}

        \item
        Damit der Ballon auf einer bestimmten H\"ohe bleibt,
        muss das Gewicht der Au\ss enluft
        genauso gro\ss\ sein wie das Gesamtgewicht des Ballons.
    \end{itemize}

    Wir fassen zusammen, dass sich das Gesamtgewicht des Ballons
    auf einer H\"ohe~\( h \) berechnet als
    \begin{align*}
        m \paren{r, h} &
        = m \paren{R} + V \paren{R} \cdot \rho_i \paren{R, h}
        = \paren[\Big]{2200 + \frac{3 \pi R^2}{5 \m[2]}} \kg
        + 2^{-\frac{2h}{11 \km}} \frac{4 R^3 \pi}{3} \frac{13 \kg}{10 \m[3]}
            \frac
                {1}
                {1 + \frac{1 \km \cdot 10^6 \m[2]}{R^2 \paren{5763 \km - 130h}}}
        \\ &
        = 2200 \kg + \frac{3 R^2}{5 \m[2]} \pi \kg
            + 4^{1 - \frac{h}{11 \km}} \frac{13 R^3}{30 \m[3]} \cdot \frac
                {1}
                {
                    1 + \frac
                        {1 \km \cdot 10^6 \m[2]}
                        {R^2 \paren{5763 \km - 130h}}
                } \pi \kg
    \end{align*}
    Gleichzeitig ist das Gewicht der Au\ss enluft auf H\"ohe~\( h \),
    die durch den Ballon verdr\"angt werden soll,
    gleich
    \begin{equation*}
        M \paren{R, h}
        \coloneqq V \paren{R} \cdot \rho \paren{h}
        = \frac{4 R^3 \pi}{3}
            \cdot 4^{-\frac{h}{11 \km}} \frac{13 \kg}{10 \m[3]}
        = 4^{1 - \frac{h}{11 \km}} \frac{13 R^3}{30 \m[3]} \pi \kg
    \end{equation*}

    Wir kennen die H\"ohe, die Theobald mindestens erreichen will,
    und k\"onnen diese Information einsetzen.
    \begin{align*}
        M \paren{R, 4 \km} &
        = 4^{1 - \frac{4 \km}{11 \km}} \frac{13 R^3}{30 \m[3]} \pi \kg
        = 4^{\frac{7}{11}} \frac{13 R^3}{30 \m[3]} \pi \kg
        \eqqcolon \widetilde{M} \paren{R}
        \displaybreak[0] \\
        m \paren{r, 4 \km} &
        = 2200 \kg + \frac{3 R^2}{5 \m[2]} \pi \kg
            + 4^{1 - \frac{4 \km}{11 \km}}
                \frac{13 R^3}{30 \m[3]} \cdot \frac
                    {1}
                    {
                        1 + \frac
                            {1 \km \cdot 10^6 \m[2]}
                            {R^2 \paren{5763 \km - 130 \cdot 4 \km}}
                    }
                \pi \kg
        \\ &
        = 2200 \kg + \frac{3 R^2}{5 \m[2]} \pi \kg
            + 4^{\frac{7}{11}} \frac{13 R^3}{30 \m[3]}
                \cdot \frac{1}{1 + \frac{10^6 \m[2]}{5243 R^2}}
            \pi \kg
        \\ &
        = 2200 \kg + \frac{3 R^2}{5 \m[2]} \pi \kg
            + \widetilde{M} \paren{R} \frac{1}{1 + \frac{10^6 \m[2]}{5243 R^2}}
        \eqqcolon \widetilde{m} \paren{R}
    \end{align*}
    \newcommand\tempR{\widetilde{R}}%
    Damit der Ballon auf der gew\"unschten H\"ohe bleibt,
    muss gelten dass~\( \widetilde{M} \paren{R} = \widetilde{m} \paren{R} \).
    Das verwenden wir, um~\( R \) zu bestimmen.
    Des Weiteren bestimmen wir dass~\( \tempR \coloneqq \frac{R}{1 \m} \)
    um die Rechnung zu vereinfachen.
    \begin{align*}
        0 &
        = \widetilde{m} \paren{R} - \widetilde{M} \paren{R}
        = 2200 \kg + \frac{3 \tempR^2 \m[2]}{5 \m[2]} \pi \kg
            + \widetilde{M} \paren{R}
                \paren[\Big]{
                    \frac{1}{1 + \frac{10^6 \m[2]}{5243 \tempR^2 \m[2]}} - 1
                }
        \\ &
        = 2200 \kg + \frac{3 \tempR^2}{5} \pi \kg
            - 4^{\frac{7}{11}} \frac{13 \tempR^3}{30}
                \cdot \frac{10^6}{5243 \tempR^2 + 10^6} \pi \kg
        \\
        2200 &
        = 4^{\frac{7}{11}}
                \frac{13 \cdot 10^6}{30 \paren{5243 \tempR^2 + 10^6}} \pi \tempR^3
            - \frac{3}{5} \pi \tempR^2
    \end{align*}
\end{document}
