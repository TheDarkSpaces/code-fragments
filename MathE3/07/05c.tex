\documentclass[../full]{subfiles}


\newcommand\OrthoVect[2][\omega]{
    #2_{\perp, #1}
}

\begin{document}
    \MainEx[Tr\"agheit von Fl\"achen]{4}

    Die Tr\"agheit einer Fl\"ache~\( F \in \RealNums^n \)
    mit Fl\"achendichte~\( 1 \)
    bez\"uglich der Drehung um eine Achse~\( \omega \in \RealNums^n \)
    mit~\( \norm{\omega} = 1 \)
    berechnet sich mittels der orthogonalen Projektion der Ortsvektoren
    auf eine zu~\( \omega \) orthogonale Ebene.
    \begin{gather*}
        \OrthoVect{x} = x - \InnerProd{x}{\omega} \omega
        \\
        I \paren{F, \omega} = \int_F \norm{\OrthoVect{x}}^2 \dF
    \end{gather*}
    F\"ur Vektoren aus~\( \RealNums^n \) gilt mit~\( \alpha \in \RealNums \)
    \begin{align*}
        \norm{x + \alpha y}^2 &
        = \InnerProd{x + \alpha y}{x + \alpha y}
        = \InnerProd{x}{x}
            + 2 \alpha \InnerProd{x}{y}
            + \alpha^2 \InnerProd{y}{y}
        = \norm{x}^2 + 2 \alpha \InnerProd{x}{y} + \alpha^2 \norm{y}^2
        \displaybreak[0] \\
        \norm{\OrthoVect{x}}^2 &
        = \InnerProd[\big]
            {x - \InnerProd{x}{\omega} \omega}{x - \InnerProd{x}{\omega} \omega}
        = \norm{x}^2
            - 2 \InnerProd{x}{\omega} \InnerProd{x}{\omega}
            + \InnerProd{x}{\omega}^2 \norm{\omega}^2
        \\ &
        = \norm{x}^2
            - 2 \InnerProd{x}{\omega}^2
            + \InnerProd{x}{\omega}^2 \cdot 1
        = \norm{x}^2 - \InnerProd{x}{\omega}^2
    \end{align*}

    Wir betrachten die Oberfl\"ache~\( F \) eines Torus in~\( \RealNums^3 \)
    mit gegebenem Au\ss enradius~\( R \) und Innenradius~\( r \),
    wobei gilt dass~\( 0 < r < R \).
    Die Fl\"ache ist parametrisiert durch
    \begin{gather*}
        \Phi \colon \IntervalCO{0}{2 \pi}^2 \to \RealNums^3
        , \enspace
        \begin{pmatrix} \varphi \\ \psi \end{pmatrix}
            \mapsto
            R \begin{pmatrix} \cos \varphi \\ \sin \varphi \\ 0 \end{pmatrix}
                + r \begin{pmatrix}
                    \cos \paren{\psi} \cos \paren{\varphi} \\
                    \cos \paren{\psi} \sin \paren{\varphi} \\
                    \sin \paren{\psi}
                \end{pmatrix}
            = \begin{pmatrix}
                R \cos \paren{\varphi}
                    + r \cos \paren{\psi} \cos \paren{\varphi} \\
                R \sin \paren{\varphi}
                    + r \cos \paren{\psi} \sin \paren{\varphi} \\
                r \sin \paren{\psi}
            \end{pmatrix}
        \displaybreak[0] \\
        \TotalD{\Phi}{\varphi, \psi}
        = \begin{pmatrix}
            -R \sin \paren{\varphi} - r \cos \paren{\psi} \sin \paren{\varphi}
                & -r \sin \paren{\psi} \cos \paren{\varphi}
            \\
            \phantom{-}
            R \cos \paren{\varphi} + r \cos \paren{\psi} \cos \paren{\varphi}
                & -r \sin \paren{\psi} \sin \paren{\varphi}
            \\
            0   & r \cos \paren{\psi}
        \end{pmatrix}
        \displaybreak[0] \\
        \det \paren[\Big]{G \paren[\big]{\TotalD{\Phi}{\varphi, \psi}}}
        = \begin{vmatrix} \paren{R + r \cos \psi}^2 & 0 \\ 0 & r^2 \end{vmatrix}
        = r^2 \paren{R + r \cos \psi}^2
    \end{gather*}

    Wir erinnern uns dass
    \begin{gather*}
        1 - \sin^2 \alpha
        = \cos^2 \alpha
        = \cos \paren{\alpha} \cos \paren{\alpha}
        = \frac{\cos \paren{\alpha + \alpha} + \cos \paren{\alpha - \alpha}}{2}
        = \frac{1 + \cos 2 \alpha}{2}
    \end{gather*}
    Uns interessiert die Tr\"agheit gegen\"uber einer Drehung um~\(
        \omega \coloneqq \begin{psmallmatrix} 0 \\ 0 \\ 1 \end{psmallmatrix}
    \).
    \begin{align*}
        I \paren{F, \omega} &
        = \int_F \norm{\OrthoVect{x}}^2 \dF
        = \int_F
            \norm[\Bigg]{
                \begin{pmatrix}
                    R \cos \paren{\varphi}
                        + r \cos \paren{\psi} \cos \paren{\varphi} \\
                    R \sin \paren{\varphi}
                        + r \cos \paren{\psi} \sin \paren{\varphi} \\
                    0
                \end{pmatrix}
            }^2
            \sqrt{\det \paren[\Big]{
                G \paren[\big]{\TotalD{\Phi}{\varphi, \psi}}
            }}
        \dx[\paren{\varphi, \psi}]
        \\ &
        = \int_F
            r \paren{R + r \cos \psi} \paren{R + r \cos \psi}^2
        \dx[\paren{\varphi, \psi}]
        = 2r \pi \int_0^{2 \pi}
            R^3 + 3 R^2 r \cos \psi + 3 R r^2 \cos^2 \psi + r^3 \cos^3 \psi
        \dx[\psi]
        \displaybreak[0] \\ &
        = 4 R^3 r \pi^2
            + 6R r^3 \pi \int_0^{2 \pi} \cos^2 \psi \dx[\psi]
            + 2 r^4 \pi \int_0^{2 \pi} \cos^3 \psi \dx[\psi]
        \\ &
        = 4 R^3 r \pi^2
            + 3R r^3 \pi \int_0^{2 \pi} 1 + \cos 2 \psi \dx[\psi]
            + 2 r^4 \pi \int_0^{2 \pi}
                \cos \paren{\psi} \paren[\big]{1 - \sin^2 \paren{\psi}}
            \dx[\psi]
        \displaybreak[0] \\ &
        = 4 R^3 r \pi^2 + 6 R r^3 \pi^2
            + 3R r^3 \pi \int_0^{2 \pi} \cos 2 \psi \dx[\psi]
            + 2 r^4 \pi \int_0^0 1 - u^2 \dx[u]
        \\ &
        = 4 R^3 r \pi^2 + 6 R r^3 \pi^2
        = 2 R r \pi^2 \paren{2 R^2 + 3 r^2}
    \end{align*}
\end{document}
