\documentclass[../full]{subfiles}


\begin{document}
    \MainEx[Volumen und Schwerpunkt von Fl\"achen]{4}

    Wir betrachtenden den Rand~\( \partial K \) zum K\"orper
    \begin{equation*}
        K = \set[\big]{
            x \in \RealNums^3
            \given
            x_3 \in \IntervalClosed{0}{1}
                , \enspace
                x_1^2 + x_2^2 \leq 1 - x_3
        }
    \end{equation*}
    Wir unterteilen~\( \partial K \) in zwei Fl\"achen
    die wir leicht parametrisieren k\"onnen
    \begin{align*}
        F_1 &
        = \set[\big]{x \in \RealNums^3 \given \norm{x} = 1, \enspace x_3 = 0},
            &
            \Phi_1 &
            \colon \IntervalClosed{0}{1} \times \IntervalCO{0}{2 \pi}
                \to \RealNums^3
            , \enspace
            \begin{pmatrix} r \\ \varphi \end{pmatrix}
                \mapsto \begin{pmatrix}
                    r \cos \varphi \\ r \sin \varphi \\ 0
                \end{pmatrix}
        \displaybreak[0] \\
        F_2 &
        = \set[\big]{
            x \in \RealNums^3
            \given
            x_3 \in \IntervalClosed{0}{1} , \enspace x_1^2 + x_2^2 = 1 - x_3
        },
            &
            \Phi_2 &
            \colon \IntervalCO{0}{2 \pi} \times \IntervalClosed{0}{1}
                \to \RealNums^3
            , \enspace
            \begin{pmatrix} \varphi \\ z \end{pmatrix}
                \mapsto \begin{pmatrix}
                    \sqrt{1 - z} \cos \varphi \\ \sqrt{1 - z} \sin \varphi \\ z
                \end{pmatrix}
    \end{align*}
    F\"ur diese Parametrisierung ergibt sich
    \begin{align*}
        \int_{F_1} 1 \dF &
        = \int_{F_1} \norm[\Bigg]{
                \begin{pmatrix}
                    \cos \varphi \\ \sin \varphi \\ 0
                \end{pmatrix}
                \times
                \begin{pmatrix}
                    -r \sin \varphi \\ \phantom{-}r \cos \varphi \\ 0
                \end{pmatrix}
            }
        \dx[\paren{r, \varphi}]
        = \int_{F_1}
            \norm[\Bigg]{\begin{pmatrix} 0 \\ 0 \\ r \end{pmatrix}}
        \dx[\paren{r, \varphi}]
        = 2 \pi \int_0^1 r \dx[r]
        = \pi
        \displaybreak[0] \\
        \int_{F_2} 1 \dF &
        = \int_{F_2}
            \norm[\Bigg]{
                \begin{pmatrix}
                    -\sqrt{1 - z} \sin \varphi \\
                    \phantom{-} \sqrt{1 - z} \cos \varphi \\
                    0
                \end{pmatrix}
                \times
                \begin{pmatrix}
                    \frac{-1}{2 \sqrt{1 - z}} \cos \varphi \\
                    \frac{-1}{2 \sqrt{1 - z}} \sin \varphi \\
                    1
                \end{pmatrix}
            }
        \dx[\paren{\varphi, z}]
        = \int_{F_2}
            \norm[\Bigg]{
                \begin{pmatrix}
                    \sqrt{1 - z} \cos \varphi \\
                    \sqrt{1 - z} \sin \varphi \\
                    \frac{1}{2}
                \end{pmatrix}
            }
        \dx[\paren{\varphi, z}]
        \\ &
        = 2 \pi \int_0^1 \frac{\sqrt{5 - 4z}}{2} \dx[z]
        % = \frac{\pi}{4} \int_1^0 -4 \sqrt{5 - 4z} \dx[z]
        = \frac{\pi}{4} \int_1^5 \sqrt{u} \dx[u]
        = \frac{\pi}{6} \FunctionBounds{\sqrt{u^3}}{1}{5}
        = \frac{5 \sqrt{5} - 1}{6} \pi
    \end{align*}
    Wir wissen dass~\( \partial K = F_1 \cup F_2 \)
    und dass~\( \mu \paren{F_1 \cap F_2} = 0 \).
    \begin{equation*}
        \volume{\partial K}
        = \volume{F_1} + \volume{F_2}
        = \frac{5 \paren{1 + \sqrt{5}}}{6} \pi
    \end{equation*}

    Weiterhin gilt
    \begin{align*}
        \int_{F_1} x \dF &
        = \int_{F_1}
            \begin{pmatrix} r \cos \varphi \\ r \sin \varphi \\ 0 \end{pmatrix}
            \norm[\Bigg]{ \begin{pmatrix} 0 \\ 0 \\ r \end{pmatrix} }
        \dx[\paren{r, \varphi}]
        = \int_{F_1}
            r^2 \begin{pmatrix} \cos \varphi \\ \sin \varphi \\ 0 \end{pmatrix}
        \dx[\paren{r, \varphi}]
        = \begin{pmatrix} 0 \\ 0 \\ 0 \end{pmatrix}
        \displaybreak[0] \\
        \int_{F_2} x \dF &
        = \int_{F_2}
            \begin{pmatrix}
                \sqrt{1 - z} \cos \varphi \\ \sqrt{1 - z} \sin \varphi \\ z
            \end{pmatrix}
            \norm[\Bigg]{
                \begin{pmatrix}
                    \sqrt{1 - z} \cos \varphi \\
                    \sqrt{1 - z} \sin \varphi \\
                    \frac{1}{2}
                \end{pmatrix}
            }
        \dx[\paren{\varphi, z}]
        \equiv \int_{F_2}
            \frac{\sqrt{5 - 4z}}{2} \begin{pmatrix} 0 \\ 0 \\ z \end{pmatrix}
        \dx[\paren{\varphi, z}]
        \\ &
        = \pi \StandardVect{3}
        \int_0^1 z \sqrt{5 - 4z} \dx[z]
        = \frac{\pi}{4} \StandardVect{3}
        \int_0^1 \paren{4z + 5 - 5} \sqrt{5 - 4z} \dx[z]
        = \frac{\pi}{16} \StandardVect{3}
        \int_5^1 \paren{u - 5} \sqrt{u} \dx[u]
        \\ &
        = \frac{\pi}{16} \StandardVect{3}
        \paren[\Bigg]{
            5 \int_1^5 \sqrt{u} \dx[u] - \int_1^5 \sqrt{u^3} \dx[u]
        }
        = \frac{\pi}{16} \StandardVect{3}
        \paren[\Bigg]{
            \frac{10}{3} \FunctionBounds{\sqrt{u^3}}{1}{5}
            - \frac{2}{5} \FunctionBounds{\sqrt{u^5}}{1}{5}
        }
        \\ &
        = \frac{\pi}{15 \cdot 16} \StandardVect{3}
        \paren[\big]{50 \paren{5 \sqrt{5} - 1} - 6 \paren{25 \sqrt{5} - 1}}
        = \frac{100 \sqrt{5} - 44}{15 \cdot 16} \pi \StandardVect{3}
        = \frac{25 \sqrt{5} - 11}{60} \pi \StandardVect{3}
    \end{align*}
    Damit berechnet sich der Fl\"achenschwerpunkt zu
    \begin{equation*}
        x_S = \frac{1}{\volume{\partial K}} \int_{\partial K} x \dF
        = \frac{6}{5 \pi \paren{1 + \sqrt{5}}}
        \paren[\Big]{\int_{F_1} x \dF + \int_{F_2} x \dF}
        = \frac{25 \sqrt{5} - 11}{50 \paren{1 + \sqrt{5}}} \StandardVect{3}
    \end{equation*}
\end{document}
