\documentclass[../full]{subfiles}


\begin{document}
    \MainEx[Holomorphie Reellwertiger Funktionen]{2}

    Es sei~\( f \) holomorph.
    \begin{gather*}
        \Phi \colon \ComplexNums \to \RealNums^2
        , \enspace
        z = x + y \iComplex \mapsto \begin{pmatrix} x \\ y \end{pmatrix}
        , \qquad
        f \colon \ComplexNums \to \ComplexNums
        , \enspace
        \Phi \paren[\big]{f \paren{x + y \iComplex}}
        = \begin{pmatrix} u \paren{x, y} \\ 0 \end{pmatrix}
        \eqqcolon \begin{pmatrix} u \paren{x, y} \\ v \paren{x, y} \end{pmatrix}
        \\
        \begin{rcases}
            \partial_x u \paren{x, y} = \phantom{-}\partial_y v \paren{x, y} = 0
            \\
            \partial_y u \paren{x, y} = -\partial_x v \paren{x, y} = 0
        \end{rcases}
        u \paren{x, y} = c \enspace \in \RealNums
    \end{gather*}
    Es folgt dass \"uberall~\( f' \paren{z} = 0 \) gelten muss.
\end{document}
