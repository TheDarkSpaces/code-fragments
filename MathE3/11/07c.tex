\documentclass[../full]{subfiles}


\def\arcsinh{\arsinh}
\def\arccosh{\arcosh}

\newenvironment{FunctionLevels}[3]
    {%
        \begin{itemize}%
            % save the formulae
            % https://tex.stackexchange.com/questions/116823/
            % Partially expanding a command
            \edef\AbsFunc {\unexpanded\expandafter{#1}}%
            \edef\RealFunc{\unexpanded\expandafter{#2}}%
            \edef\ImagFunc{\unexpanded\expandafter{#3}}%
    }  {\end{itemize}}

\newenvironment{ResultLevel}[1]
    {\item Niveau~\( c = #1 \) \begin{itemize}\edef\CurrentLevel{#1}}
    {\end{itemize}}

\NewEnviron{AbsLevel}[1][gather*]{%
    \item \( \abs[\big]{f \paren{x + y \iComplex}} = \AbsFunc = \CurrentLevel \)
    \begin{#1} \BODY \end{#1}%
}
\NewEnviron{RealLevel}[1][gather*]{%
    \item \( \RealPart f \paren{x + y \iComplex} = \RealFunc = \CurrentLevel \)
    \begin{#1} \BODY \end{#1}%
}
\NewEnviron{ImagLevel}[1][gather*]{%
    \item \( \ImagPart f \paren{x + y \iComplex} = \ImagFunc = \CurrentLevel \)
    \begin{#1} \BODY \end{#1}%
}

\begin{document}
    \MainEx[Niveaulinien in der Gauss-Ebene]{6}
    \label{task:function_levels:complex}


    \PartEx[Sinus]{3}
    \label{task:function_levels:complex:Sine}

    \begin{figure}
        \centering
        \HomeworkPart[Plots]{Sine}

        \caption*{Skizzen zu \hyperref[task:function_levels:complex:Sine]{%
            \ref*{task:function_levels:complex}%
            ~\ref*{task:function_levels:complex:Sine}%
        }, mit~\( f \paren{z} = \sin z \)}
        \label{task:function_levels:complex:Sine:Plots}
    \end{figure}

    Skizzen befinden sich auf~\hyperref[task:function_levels:complex:Sine:Plots]{%
        Seite~\pageref*{task:function_levels:complex:Sine:Plots}%
    }.

    Wir definieren die Ausdr\"ucke
    \begin{align*}
        \sinh x \coloneqq \frac{\eEuler^x - \eEuler^{-x}}{2}
        & , \qquad
        \cosh x \coloneqq \frac{\eEuler^x + \eEuler^{-x}}{2}
        \\
        \shortintertext{
            Offensichtlich gilt damit, analog zu~\( \sin \) und~\( \cos \)
        }
        \sinh \paren{-x} = \frac{\eEuler^{-x} - \eEuler^x}{2} = -\sinh x
        & , \qquad
        \cosh \paren{-x} = \frac{\eEuler^{-x} + \eEuler^x}{2} = \cosh x
    \end{align*}

    Wir bemerken dass~\( \sinh \) auf ganz~\( \RealNums \)
    streng monoton w\"achst.
    Damit existiert im Reellen eine Umkehrfunktion,
    die wir~\( \arsinh \) nennen.
    \begin{gather*}
        a = \sinh b = \frac{\eEuler^b - \eEuler^{-b}}{2}
            = \frac{\eEuler^b - \frac{1}{\eEuler^b}}{2}
        \enspace \Leftrightarrow \enspace
        0 = \paren{\eEuler^b}^2 - 2a \eEuler^b - 1
        \enspace \Leftrightarrow \enspace
        \eEuler^b = \frac{2a \pm 2 \sqrt{a^2 + 1}}{2} = a \pm \sqrt{a^2 + 1}
        \\
        \shortintertext{
            Wir wissen, dass~\( \sqrt{a^2 + 1} > \sqrt{a^2} = \abs{a} \).
            Weil aber~\( \eEuler^b > 0 \) gilt,
            f\"allt die L\"osung~\( \eEuler^b = a - \sqrt{a^2 + 1} < 0 \) weg.
        }
        \eEuler^b = a + \sqrt{a^2 + 1}
        \enspace \Leftrightarrow \enspace
        b = \arsinh \paren{\sinh b}
            = \arsinh a = \ln \paren[\big]{a + \sqrt{a^2 + 1}}
        \displaybreak[0] \\
        \shortintertext{
            Aus der ungeraden Parit\"at von~\( \sinh \)
            folgt die ungerade Parit\"at von~\( \arsinh \).
        }
        \arsinh \paren{-\sinh x} = \arsinh \paren[\big]{\sinh \paren{-x}}
            = -x = -\arsinh \paren{\sinh x}
    \end{gather*}

    Ebenso w\"achst~\( \cosh \) streng monoton auf~\(\RealNumsZeroPos \).
    Damit existiert eine Umkehrfunktion auf diese Werte,
    die wir~\( \arcosh \) nennen.
    \begin{gather*}
        a = \cosh b = \frac{\eEuler^b + \eEuler^{-b}}{2}
            = \frac{\eEuler^b + \frac{1}{\eEuler^b}}{2}
        \enspace \Leftrightarrow \enspace
        0 = \paren{\eEuler^b}^2 - 2a \eEuler^b + 1
        \enspace \Leftrightarrow \enspace
        \eEuler^b = \frac{2a \pm 2 \sqrt{a^2 - 1}}{2} = a \pm \sqrt{a^2 - 1}
        \\
        \shortintertext{
            Weil~\( \cosh b = a \) streng monoton w\"achst,
            muss mit~\( b \) auch~\( a \) gr\"o\ss er werden.
            W\"ahrend aber~\( \eEuler^b \) streng monoton w\"achst,
            ist das Ergebnis~\( a - \sqrt{a^2 - 1} \) streng monoton fallend,
            also f\"allt diese L\"osung weg.
        }
        \eEuler^b = a + \sqrt{a^2 - 1}
        \enspace \Leftrightarrow \enspace
        b = \arcosh \paren{\cosh b}
            = \arcosh a = \ln \paren[\big]{a + \sqrt{a^2 - 1}}
        \displaybreak[0] \\
        \shortintertext{
            Weil~\( \cosh x \) von gerader Parit\"at ist,
            ist~\( \arcosh \) nur bis auf das Vorzeichen bestimmt.
        }
        x = \cosh \paren{\arcosh x} = \cosh \paren{-\arcosh x}
    \end{gather*}

    Wir erkennen dass
    \begin{gather*}
        \cosh^2 x = \frac{\eEuler^{2x} + \eEuler^{-2x} + 2}{4}
        = \frac{\cosh 2x + 1}{2} = \frac{\cosh 2x - 1}{2} + 1
        = \frac{\eEuler^{2x} + \eEuler^{-2x} - 2}{4} + 1 = \sinh^2 x + 1
    \end{gather*}

    Mit diesem Wissen k\"onnen wir die komplexe Sinus-Funktion beschreiben:
    \begin{align*}
        f \paren{x + y \iComplex} &
        = \sin \paren{x + y \iComplex}
        = \frac{\eEuler^y + \eEuler^{-y}}{2} \sin x
            + \frac{\eEuler^y - \eEuler^{-y}}{2} \iComplex \cos x
        = \cosh \paren{y} \sin \paren{x}
            + \iComplex \sinh \paren{y} \cos \paren{x}
        \displaybreak[0] \\
        \abs[\big]{f \paren{x + y \iComplex}} &
        = \sqrt{
            \cosh^2 \paren{y} \sin^2 \paren{x}
            + \sinh^2 \paren{y} \cos^2 \paren{x}
        }
        = \sqrt{
            \sinh^2 \paren{y} \cos^2 \paren{x}
            + \paren[\big]{\sinh^2 \paren{y} + 1} \sin^2 \paren{x}
        }
        \\ &
        = \sqrt{\sinh^2 y + \sin^2 x}
    \end{align*}

    \begin{FunctionLevels}
                {\sqrt{\sinh^2 y + \sin^2 x}}
                {\cosh \paren{y} \sin \paren{x}}
                {\sinh \paren{y} \cos \paren{x}}
        \begin{ResultLevel}{-1}
            \begin{AbsLevel}[equation*]
                x, y \in \emptyset
            \end{AbsLevel}

            \begin{RealLevel}
                \cosh y = \frac{-1}{\sin x}
                \enspace \Rightarrow \enspace
                y = \pm \arccosh \paren[\Big]{\frac{-1}{\sin x}}
                \\
                \forall y \in \RealNums \colon \cosh y > 0
                \enspace \Rightarrow \enspace
                \sin x < 0
                \enspace \Leftrightarrow \enspace
                x \in \paren[\Big]
                    {\bigcup_{n \in \IntNums} \IntervalOpen{2n - 1}{2n}}
                    \pi
                \\
                \lim_{x \toGT -\pi} \arccosh \paren[\Big]{\frac{-1}{\sin x}}
                = \lim_{x \toLT 0} \arccosh \paren[\Big]{\frac{-1}{\sin x}}
                = \infty
            \end{RealLevel}

            \begin{ImagLevel}
                \sinh y = \frac{-1}{\cos x}
                \enspace \Leftrightarrow \enspace
                y = \arcsinh \paren[\Big]{\frac{-1}{\cos x}}
                = -\arcsinh \paren[\Big]{\frac{1}{\cos x}}
                \\
                x \in \RealNums \setminus \frac{\paren{2 \IntNums + 1} \pi}{2}
                \\
                \lim_{\epsilon^+ \toGT 0} \paren[\big]{
                    -\arcsinh \paren[\Big]{
                        \frac{1}{\cos \paren{\frac{\pi}{2} \pm \epsilon^+}}
                    }
                }
                = -\lim_{\epsilon^+ \toGT 0} \paren[\big]{
                    -\arcsinh \paren[\Big]{
                        \frac{1}{\cos \paren{\frac{3 \pi}{2} \pm \epsilon^+}}
                    }
                }
                = \pm \infty
            \end{ImagLevel}
        \end{ResultLevel}

        \begin{ResultLevel}{0}
            \begin{AbsLevel}
                \sinh^2 y + \sin^2 x = 0
                \enspace \Leftrightarrow \enspace
                \sinh^2 y = -\sin^2 x
                \enspace \Rightarrow \enspace
                \sinh^2 y = 0
                \enspace \Leftrightarrow \enspace
                y = 0
                \\
                \sinh^2 y = 0 = -\sin^2 x
                \enspace \Rightarrow \enspace
                \sin x = 0
                \enspace \Leftrightarrow \enspace
                x \in \IntNums \pi
            \end{AbsLevel}

            \begin{RealLevel}
                \forall y \in \RealNums \colon \cosh y > 0
                \enspace \Rightarrow \enspace
                y \in \RealNums
                \\
                \sin x = 0
                \enspace \Leftrightarrow \enspace
                x \in \IntNums \pi
            \end{RealLevel}

            \begin{ImagLevel}
                \sinh y = 0
                \enspace \Leftrightarrow \enspace
                y = 0, \enspace x \in \RealNums
                \\
                \cos x = 0
                \enspace \Leftrightarrow \enspace
                x \in \frac{\paren{2 \IntNums + 1} \pi}{2}
                    , \enspace
                    y \in \RealNums
            \end{ImagLevel}
        \end{ResultLevel}

        \begin{ResultLevel}{1}
            \begin{AbsLevel}[align*]
                    \sinh^2 y + \sin^2 x = 1 &
                    \enspace \Leftrightarrow \enspace
                    \sinh^2 y = 1 - \sin^2 x = \cos^2 x
                    \enspace \Leftrightarrow \enspace
                    \sinh y = \pm \cos x
                    \\ &
                    \enspace \Leftrightarrow \enspace
                    y = \arcsinh \paren{\pm \cos x} = \pm \arcsinh \paren{\cos x}
            \end{AbsLevel}

            \begin{RealLevel}
                \cosh y = \frac{1}{\sin x}
                \enspace \Rightarrow \enspace
                y = \pm \arccosh \paren[\Big]{\frac{1}{\sin x}}
                \\
                \forall y \in \RealNums \colon \cosh y > 0
                \enspace \Rightarrow \enspace
                \sin x > 0
                \enspace \Leftrightarrow \enspace
                x \in \paren[\Big]
                    {\bigcup_{n \in \IntNums} \IntervalOpen{2n}{2n + 1}}
                    \pi
                \\
                \lim_{x \toGT 0} \arccosh \paren[\Big]{\frac{1}{\sin x}}
                = \lim_{x \toLT \pi} \arccosh \paren[\Big]{\frac{1}{\sin x}}
                = \infty
            \end{RealLevel}

            \begin{ImagLevel}
                \sinh y = \frac{1}{\cos x}
                \enspace \Leftrightarrow \enspace
                y = \arcsinh \paren[\Big]{\frac{1}{\cos x}}
                \\
                \cos x \neq 0
                \enspace \Rightarrow \enspace
                x \in \RealNums \setminus \frac{\paren{2 \IntNums + 1} \pi}{2}
                \\
                \lim_{\epsilon^+ \toGT 0} \paren[\big]{
                    \arcsinh \paren[\Big]{
                        \frac{1}{\cos \paren{\frac{\pi}{2} \pm \epsilon^+}}
                    }
                }
                = -\lim_{\epsilon^+ \toGT 0} \paren[\big]{
                    \arcsinh \paren[\Big]{
                        \frac{1}{\cos \paren{\frac{3 \pi}{2} \pm \epsilon^+}}
                    }
                }
                = \mp \infty
            \end{ImagLevel}
        \end{ResultLevel}
    \end{FunctionLevels}


    \PartEx[Bruch]{3}
    \label{task:function_levels:complex:Fraction}

    \begin{figure}
        \centering
        \HomeworkPart[Plots]{Fraction}

        \caption*{Skizzen zu \hyperref[task:function_levels:complex:Fraction]{%
            \ref*{task:function_levels:complex}%
            ~\ref*{task:function_levels:complex:Fraction}%
        }, mit~\( f \paren{z} = \frac{1}{z} \)}
        \label{task:function_levels:complex:Fraction:Plots}
    \end{figure}

    Skizzen befinden sich
    auf~\hyperref[task:function_levels:complex:Fraction:Plots]{%
        Seite~\pageref*{task:function_levels:complex:Fraction:Plots}%
    }.

    \begin{gather*}
        f \paren{x + y \iComplex}
        = \frac{1}{x + y \iComplex}
        = \frac{x}{x^2 + y^2} - \frac{y}{x^2 + y^2} \iComplex
        \\
        \abs[\big]{f \paren{z}}
        = \abs[\big]{f \paren{x + y \iComplex}}
        = \sqrt{
            \paren[\Big]{\frac{x}{x^2 + y^2}}^2
            + \paren[\Big]{\frac{-y}{x^2 + y^2}}^2
        }
        = \sqrt{\frac{x^2 + y^2}{\paren[\big]{x^2 + y^2}^2}}
        = \frac{1}{\sqrt{x^2 + y^2}}
        = \frac{1}{\abs{z}}
    \end{gather*}

    \begin{FunctionLevels}
                {\frac{1}{\sqrt{x^2 + y^2}}}
                {\frac{x}{x^2 + y^2}}
                {-\frac{y}{x^2 + y^2}}
        \begin{ResultLevel}{-1}
            \begin{AbsLevel}[equation*]
                x, y \in \emptyset
            \end{AbsLevel}

            \begin{RealLevel}
                x < 0
                \enspace \Rightarrow \enspace
                x^2 + y^2 = -x
                \enspace \Leftrightarrow \enspace
                0 = x^2 + x + y^2
                \enspace \Leftrightarrow \enspace
                \paren[\Big]{\frac{1}{2}}^2
                = x^2 + x + \paren[\Big]{\frac{1}{2}}^2 + y^2
                = \paren[\Big]{x + \frac{1}{2}}^2 + y^2
                \\
                x < 0
                \enspace \Rightarrow \enspace
                \begin{pmatrix} x \\ y \end{pmatrix} \in \set[\Bigg]{
                    \frac{1}{2}
                    \begin{pmatrix} \cos \varphi \\ \sin \varphi \end{pmatrix}
                        - \frac{1}{2} \StandardVect{1}
                    \given
                    \varphi \in
                        \IntervalOpen[\Big]{\frac{\pi}{2}}{\frac{3 \pi}{2}}
                }
            \end{RealLevel}

            \begin{ImagLevel}
                y > 0
                \enspace \Rightarrow \enspace
                x^2 + y^2 = y
                \enspace \Leftrightarrow \enspace
                0 = x^2 + y^2 - y
                \enspace \Leftrightarrow \enspace
                \paren[\Big]{\frac{1}{2}}^2
                = x^2 + y^2 - y + \paren[\Big]{\frac{1}{2}}^2
                = x^2 + \paren[\Big]{y - \frac{1}{2}}^2
                \\
                y > 0
                \enspace \Rightarrow \enspace
                \begin{pmatrix} x \\ y \end{pmatrix} \in \set[\Bigg]{
                    \frac{1}{2}
                    \begin{pmatrix} \cos \varphi \\ \sin \varphi \end{pmatrix}
                        + \frac{1}{2} \StandardVect{2}
                    \given
                    \varphi \in \IntervalOpen[\big]{0}{\pi}
                }
            \end{ImagLevel}
        \end{ResultLevel}

        \begin{ResultLevel}{0}
            \begin{AbsLevel}[equation*]
                x, y \in \emptyset
            \end{AbsLevel}

            \begin{RealLevel}[equation*]
                x = 0
                \enspace \Rightarrow \enspace
                y \in \RealNums \setminus \set{0}
            \end{RealLevel}

            \begin{ImagLevel}[equation*]
                y = 0
                \enspace \Rightarrow \enspace
                x \in \RealNums \setminus \set{0}
            \end{ImagLevel}
        \end{ResultLevel}

        \begin{ResultLevel}{1}
            \begin{AbsLevel}[equation*]
                \frac{1}{x^2 + y^2} = 1
                \enspace \Leftrightarrow \enspace
                x^2 + y^2 = 1
                \enspace \Rightarrow \enspace
                \begin{pmatrix} x \\ y \end{pmatrix} \in \set[\Bigg]{
                    \begin{pmatrix} \cos \varphi \\ \sin \varphi \end{pmatrix}
                    \given
                    \varphi \in \IntervalCO[\big]{0}{2 \pi}
                }
            \end{AbsLevel}

            \begin{RealLevel}
                x > 0
                \enspace \Rightarrow \enspace
                x^2 + y^2 = x
                \enspace \Leftrightarrow \enspace
                0 = x^2 - x + y^2
                \enspace \Leftrightarrow \enspace
                \paren[\Big]{\frac{1}{2}}^2
                = x^2 - x + \paren[\Big]{\frac{1}{2}}^2 + y^2
                = \paren[\Big]{x - \frac{1}{2}}^2 + y^2
                \\
                x > 0
                \enspace \Rightarrow \enspace
                \begin{pmatrix} x \\ y \end{pmatrix} \in \set[\Bigg]{
                    \frac{1}{2}
                    \begin{pmatrix} \cos \varphi \\ \sin \varphi \end{pmatrix}
                        + \frac{1}{2} \StandardVect{1}
                    \given
                    \varphi \in
                        \IntervalOpen[\Big]{-\frac{\pi}{2}}{\frac{\pi}{2}}
                }
            \end{RealLevel}

            \begin{ImagLevel}
                y < 0
                \enspace \Rightarrow \enspace
                x^2 + y^2 = -y
                \enspace \Leftrightarrow \enspace
                0 = x^2 + y^2 + y
                \enspace \Leftrightarrow \enspace
                \paren[\Big]{\frac{1}{2}}^2
                = x^2 + y^2 + y + \paren[\Big]{\frac{1}{2}}^2
                = x^2 + \paren[\Big]{y + \frac{1}{2}}^2
                \\
                y < 0
                \enspace \Rightarrow \enspace
                \begin{pmatrix} x \\ y \end{pmatrix} \in \set[\Bigg]{
                    \frac{1}{2}
                    \begin{pmatrix} \cos \varphi \\ \sin \varphi \end{pmatrix}
                        - \frac{1}{2} \StandardVect{2}
                    \given
                    \varphi \in \IntervalOpen[\big]{-\pi}{0}
                }
            \end{ImagLevel}
        \end{ResultLevel}
    \end{FunctionLevels}
\end{document}
