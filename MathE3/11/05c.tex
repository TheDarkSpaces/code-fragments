\documentclass[../full]{subfiles}


\begin{document}
    \MainEx[Holomorphe Harmonie-Anforderung]{2}

    Jede holomorphe Funktion~\( f \colon \ComplexNums \to \ComplexNums \)
    muss im Realteil (und ebenso im Imagin\"arteil) harmonisch sein,
    das hei\ss t
    \begin{equation*}
        \partial_{xx} \paren{\RealPart f} \paren{x, y}
        = \partial_{yy} \paren{\RealPart f} \paren{x, y}
    \end{equation*}

    Wir betrachten eine Funktion~\( f \paren{x + y \iComplex} \)
    bei der der Realteil beschrieben wird durch~\( u \paren{x, y} \)
    \begin{gather*}
        \RealPart f \paren{x + y \iComplex}
        \coloneqq u \paren{x, y}
        = \sin x + \eEuler^y
        \\
        \partial_{xx} u \paren{x, y}
        = -\sin x
        \neq \eEuler^y
        = \partial_{yy} u \paren{x, y}
    \end{gather*}
    Da die Harmonie im Realteil nicht gegeben ist,
    kann~\( f \) nicht holomorph sein.
\end{document}
