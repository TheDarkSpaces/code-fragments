\documentclass[../../full]{subfiles}


\providecommand\ColorAbs {Cerulean}
\providecommand\ColorReal{LimeGreen}
\providecommand\ColorImag{OrangeRed}

\newcommand\xmax{2.75*pi}
\newcommand\xmin{-\xmax}
\newcommand\ymax{6.25}
\newcommand\ymin{-\ymax}

\pgfplotsset{
    lua support=false, % TODO why does the lua portion fail?
    only points/.style={
        only marks,
    },
    include points/.style={
        mark=*,
    },
    function levels/.style={
        thick,
        no marks,
    },
    abs levels/.style={
        function levels,
        \ColorAbs,
    },
    real levels/.style={
        function levels,
        \ColorReal,
    },
    imag levels/.style={
        function levels,
        \ColorImag,
    },
    % see https://tex.stackexchange.com/a/396298
    % Axis with trigonometric labels in PGFPlots
    x axis in pi/.style={
        xtick distance={pi/#1},
        xticklabel={
            \tikzmath{
                % Calculate this tick's multiple of pi/#1
                int \numorig, \gcd, \num, \denom, \absnum;
                \numorig = round(\tick*#1 / pi);
                % Calculate reduced fraction for \numorig/#1
                \gcd = gcd(\numorig, #1);
                \num = \numorig / \gcd;
                \absnum = abs(\num);
                \denom = #1 / \gcd;
                % Build label text
                if \num < 0 then {
                    let \sign = -;
                } else {
                    let \sign =;
                };
                if \absnum == 1 then {
                    let \numpi = \pi;
                } else {
                    let \numpi = \absnum\pi;
                };
                % the strut ensures that all labels have the same baseline
                % without it, the fractional entries are different
                if \denom == 1 then {
                    if \num == 0 then {
                        { \strut\( 0 \) };
                    } else {
                        { \strut\( \sign\numpi \) };
                    };
                } else {
                    { \strut\( \sign\frac{\numpi}{\denom} \) };
                    % Other style with all pi symbols same and aligned:
                    %{ \strut\( \sign\frac{\absnum}{\denom}\pi \) };
                };
            }
        },
    },
}
\pgfkeys{
    pgf/declare function={
        asinh(\x) = ln(\x + sqrt((\x)^2 + 1));
        acosh(\x) = ln(\x + sqrt((\x)^2 - 1));
        % analogous: sec(x) = 1/cos(x), cosec(x) = 1/sin(x)
        sech(\x) = reciprocal(cosh(\x));
        cosech(\x) = reciprocal(sinh(\x));
    },
}

\begin{document}
    \begin{tikzpicture}
        \begin{groupplot}[
                    group style={
                        group size=1 by 3,
                        group name=Sine Plots,
                        every plot/.append style={
                            smooth,
                            samples=401,
                        },
                    },
                    width=0.8*\textwidth,
                    height=0.3*(\textheight
                                - 2*\pgfkeysvalueof{/pgfplots/group/horizontal sep}),
                    xlabel={\( \RealPart f \)},
                    ylabel={\( \ImagPart f \)},
                    xmin=\xmin, xmax=\xmax,
                    ymin=\ymin, ymax=\ymax,
                    x axis in pi=2,
                    minor x tick num=2,
                    minor y tick num=1,
                    axis lines=center,
                ]
            % c = -1
            \nextgroupplot[
                    title={\( c = -1 \)},
                    declare function={
                        % we want to change sign when x = -pi/2
                        %                             (or a periodic multiple thereof)
                        % we observe that sin(-pi/2) is a minimum
                        % so we use sin' x = cos x
                        %           to determine whether we have passed this point
                        % a value smaller than 0 means that the point is yet to come,
                        % a value greater than 0 means that the point has passed
                        RealYSign(\x) = sign(-cos(\x r));
                        RealYRaw(\x) = acosh(-cosec(\x r));
                        RealY(\x) = RealYSign(\x) * RealYRaw(\x);
                        RealXCutoff(\y) = rad( asin(sech(\y)) );
                        RealXBoundMin(\y,\n) = (2*\n - 1)*pi + RealXCutoff(\y);
                        RealXBoundMax(\y,\n) = (2*\n + 0)*pi - RealXCutoff(\y);
                        ImagY(\x) = -asinh(sec(\x r));
                        ImagXCutoff(\y) = rad( acos(-cosech(\y)) );
                        ImagXBoundMin(\y,\n,\s) = (2*\n - 1 + \s)*pi + ImagXCutoff(\y);
                        ImagXBoundMax(\y,\n,\s) = (2*\n + 1 + \s)*pi - ImagXCutoff(\y);
                    },
                    legend columns=1,
                    legend cell align=right,
                    legend to name=Sine Legend,
                ]
            % list line styles for the legend, where no one sees them
            \addplot [only points, include points] coordinates {(\xmin-1, \ymin-1)};
                \addlegendentry{mit Punkt}
            \addplot [abs levels] coordinates {(\xmin-1, \ymin-1)};
                \addlegendentry{\( \abs[\big]{f \paren{z}} = c \)}
            \addplot [real levels] coordinates {(\xmin-1, \ymin-1)};
                \addlegendentry{\( \RealPart f \paren{z} = c \)}
            \addplot [imag levels] coordinates {(\xmin-1, \ymin-1)};
                \addlegendentry{\( \ImagPart f \paren{z} = c \)}
            % end legend
            \def\Plots#1{
                % Real
                \pgfmathsetmacro\xDMin{RealXBoundMin(\ymax+0.25, #1)}
                \pgfmathsetmacro\xDMax{RealXBoundMax(\ymax+0.25, #1)}
                \addplot [real levels, domain=\xDMin:\xDMax] { RealY(x)};
                \addplot [real levels, domain=\xDMin:\xDMax] {-RealY(x)};
                % Imag
                \pgfmathsetmacro\xDMin{ImagXBoundMin(\ymax+0.25, #1, 0)}
                \pgfmathsetmacro\xDMax{ImagXBoundMax(\ymax+0.25, #1, 0)}
                \addplot [imag levels, domain=\xDMin:\xDMax] { ImagY(x)};
                % Imag Shifted
                \pgfmathsetmacro\xDMin{ImagXBoundMin(\ymax+0.25, #1, 1)}
                \pgfmathsetmacro\xDMax{ImagXBoundMax(\ymax+0.25, #1, 1)}
                \addplot [imag levels, domain=\xDMin:\xDMax] { ImagY(x)};
            }
            \foreach \Period in {-2, ..., 1} {
                \edef\Temp{\noexpand\Plots{\Period}}
                \Temp
            }
            % c = 0
            \nextgroupplot[
                    title={\( c = 0 \)},
                ]
            \def\Plots#1{
                % Abs
                \addplot [abs levels, only points, include points] coordinates {(#1*pi, 0)};
                % Real
                \addplot [real levels] coordinates {(#1*pi, \ymin) (#1*pi, \ymax)};
                % Imag
                \addplot [imag levels] coordinates {((#1+0.5)*pi, \ymin) ((#1+0.5)*pi, \ymax)};
            }
            \foreach \Period in {-3, ..., 2} {
                \edef\Temp{\noexpand\Plots{\Period}}
                \Temp
            }
            \nextgroupplot[
                    title={\( c = 1 \)},
                    declare function={
                        % we want to change sign when x = pi/2
                        %                             (or a periodic multiple thereof)
                        % we observe that sin(pi/2) is a maximum
                        % so we use sin' x = cos x
                        %           to determine whether we have passed this point
                        % a value greater than 0 means that the point is yet to come,
                        % a value smaller than 0 means that the point has passed
                        AbsY(\x) = asinh(cos(\x r));
                        RealYSign(\x) = sign(cos(\x r));
                        RealYRaw(\x) = acosh(cosec(\x r));
                        RealY(\x) = RealYSign(\x) * RealYRaw(\x);
                        RealXCutoff(\y) = rad( asin(sech(\y)) );
                        RealXBoundMin(\y,\n) = (2*\n - 0)*pi + RealXCutoff(\y);
                        RealXBoundMax(\y,\n) = (2*\n + 1)*pi - RealXCutoff(\y);
                        ImagY(\x) = asinh(sec(\x r));
                        ImagXCutoff(\y) = rad( acos(-cosech(\y)) );
                        ImagXBoundMin(\y,\n,\s) = (2*\n - 1 + \s)*pi + ImagXCutoff(\y);
                        ImagXBoundMax(\y,\n,\s) = (2*\n + 1 + \s)*pi - ImagXCutoff(\y);
                    },
                ]
            % 13 points of interest, meaning I want 13+12*n samples, and 613 samples fits nicely
            \addplot [abs levels, domain=-3*pi:3*pi, samples=613] { AbsY(x)};
            \addplot [abs levels, domain=-3*pi:3*pi, samples=613] {-AbsY(x)};
            \def\Plots#1{
                % Real
                \pgfmathsetmacro\xDMin{RealXBoundMin(\ymax+0.25, #1)}
                \pgfmathsetmacro\xDMax{RealXBoundMax(\ymax+0.25, #1)}
                \addplot [real levels, domain=\xDMin:\xDMax] { RealY(x)};
                \addplot [real levels, domain=\xDMin:\xDMax] {-RealY(x)};
                % Imag
                \pgfmathsetmacro\xDMin{ImagXBoundMin(\ymax+0.25, #1, 0)}
                \pgfmathsetmacro\xDMax{ImagXBoundMax(\ymax+0.25, #1, 0)}
                \addplot [imag levels, domain=\xDMin:\xDMax] { ImagY(x)};
                % Imag Shifted
                \pgfmathsetmacro\xDMin{ImagXBoundMin(\ymax+0.25, #1, 1)}
                \pgfmathsetmacro\xDMax{ImagXBoundMax(\ymax+0.25, #1, 1)}
                \addplot [imag levels, domain=\xDMin:\xDMax] { ImagY(x)};
            }
            \foreach \Period in {-2, ..., 1} {
                \edef\Temp{\noexpand\Plots{\Period}}
                \Temp
            }
        \end{groupplot}
        \node at ($(Sine Plots c1r1.north east)!0.5!(Sine Plots c1r3.south east)$)
                [right=2em] {\ref{Sine Legend}};
    \end{tikzpicture}
\end{document}
