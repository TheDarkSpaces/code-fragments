\documentclass[../full]{subfiles}


\begin{document}
    \MainEx[Torus in 3D]{4}
    \label{task:torus:3d}

    Wir definieren wir einen Torus um den Ursprung~\( 0 \in \RealNums^3 \)
    mit~\( R_1 > R_2 \geq 0 \),
    parametrisiert durch
    \begin{equation*}
        \Phi \colon
            \IntervalClosed{0}{R_2}
                    \times \IntervalCO{0}{2 \pi}
                    \times \IntervalCO{0}{2 \pi}
                \to \RealNums^3,
            \enspace
            \begin{pmatrix} r \\ \varphi \\ \psi \end{pmatrix}
                \mapsto
                R_1 \begin{pmatrix}
                    \cos \varphi \\ \sin \varphi \\ 0
                \end{pmatrix}
                + r \paren[\Bigg]{
                    \cos \psi \begin{pmatrix}
                        \cos \varphi \\ \sin \varphi \\ 0
                    \end{pmatrix}
                    + \sin \psi \begin{pmatrix} 0 \\ 0 \\ 1 \end{pmatrix}
                }
    \end{equation*}


    \PartEx[Skizze und Jacobi-Matrix]{2}
    \label{task:torus:3d:sketch}

    \begin{figure}
        \centering
        \HomeworkPart[Sketch]{Torus}

        \caption*{%
            Skizze zu
            \hyperref[task:torus:3d:sketch]{%
                \ref*{task:torus:3d}%
                ~\ref*{task:torus:3d:sketch}%
            }%
        }
        \label{task:torus:3d:sketch:visual}
    \end{figure}

    \begin{gather*}
        \TotalD{\Phi}{r, \varphi, \psi}
        = \begin{pmatrix}
            \cos \paren{\psi} \cos \paren{\varphi}
                & -R_1 \sin \paren{\varphi}
                    - r \cos \paren{\psi}  \sin \paren{\varphi}
                & -r \sin \paren{\psi} \cos \paren{\varphi}
            \\
            \cos \paren{\psi} \sin \paren{\varphi}
                & \phantom{-} R_1 \cos \paren{\varphi}
                    + r \cos \paren{\psi} \cos \paren{\varphi}
                & -r \sin \paren{\psi} \sin \paren{\varphi}
            \\
            \sin \paren{\psi}
                & 0
                & r \cos \paren{\psi}
        \end{pmatrix}
        \\
        \det \paren[\big]{\TotalD{\Phi}{r, \varphi, \psi}}
        = \paren[\Big]{
            \paren[\big]{r^2 \cos \paren{\psi} + R_1 r}
            \paren[\big]{\cos^2 \paren{\psi} + \sin^2 \paren{\psi}}
        } \paren[\big]{\cos^2 \paren{\varphi} + \sin^2 \paren{\varphi}}
        = r^2 \cos \paren{\psi} + R_1 r
    \end{gather*}
    Eine Skizze befindet sich auf \hyperref[task:torus:3d:sketch:visual]{%
        Seite~\pageref*{task:torus:3d:sketch:visual}%
    }.


    \PartEx[Volumenberechnung]{2}

    Das Volumen~\( V_{R_1, R_2} \) des Torus berechnet sich durch
    \begin{equation*}
        V_{R_1, R_2}
        = \int_{r = 0}^{R_2}
            \int_{\varphi = 0}^{2 \pi}
                \int_{\psi = 0}^{2 \pi}
                    \abs[\Big]
                        {\det \paren[\big]{\TotalD{\Phi}{r, \varphi, \psi}}}
                \dx[\psi]
            \dx[\varphi]
        \dx[r]
        = \int_{r = 0}^{R_2}
            \int_{\varphi = 0}^{2 \pi}
                \int_{\psi = 0}^{2 \pi}
                    \abs[\big]{R_1 r + r^2 \cos \psi}
                \dx[\psi]
            \dx[\varphi]
        \dx[r]
    \end{equation*}
    Weil~\( \cos \psi \in \IntervalClosed{-1}{1} \) gilt,
    wissen wir aus der Eigenschaft~\( R_1 > R_2 \),
    dass
    \begin{gather*}
        \forall r \in \IntervalClosed{0}{R_2} \colon R_1 r > r^2
        \enspace \Rightarrow \enspace
        \forall \psi \in \IntervalClosed{0}{2 \pi} \colon
            0 = r^2 - r^2 < R_1 r - r^2 \leq R_1 r + r^2 \cos \psi
        \\
        \det \paren[\big]{\TotalD{\Phi}{r, \varphi, \psi}}
        = R_1 r + r^2 \cos \psi
        > 0
        \enspace \Rightarrow \enspace
        \abs[\Big]{\det \paren[\big]{\TotalD{\Phi}{r, \varphi, \psi}}}
        = \det \paren[\big]{\TotalD{\Phi}{r, \varphi, \psi}}
    \end{gather*}

    Damit ergibt sich
    \begin{align*}
        V_{R_1, R_2} &
        = \int_{r = 0}^{R_2}
            \int_{\varphi = 0}^{2 \pi}
                \int_{\psi = 0}^{2 \pi} R_1 r + r^2 \cos \psi \dx[\psi]
            \dx[\varphi]
        \dx[r]
        = 2 \pi \int_{r = 0}^{R_2}
            2 R_1 r \pi + r^2 \int_{\psi = 0}^{2 \pi}
                \cos \psi
            \dx[\psi]
        \dx[r]
        \\ &
        = 2 \pi \int_0^{R_2} 2 R_1 r \pi + r^2 \cdot 0 \dx[r]
        = 4 R_1 \pi^2 \int_0^{R_2} r \dx[r]
        = 4 R_1 \pi^2 \FunctionBounds{\frac{r^2}{2}}{0}{R_2}
        = 2 R_1 R_2^2 \pi^2
        = \paren{2 R_1 \pi} \paren{R_2^2 \pi}
    \end{align*}
\end{document}
