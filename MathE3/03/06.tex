\documentclass[../full]{subfiles}


\begin{document}
    \MainEx[Volumen von Ellipsoiden]{2}


    \PartEx[Ellipsen]{1}

    Zu~\( a, b > 0 \) beschreiben wir
    eine Ellipse mit Halbachsen~\( a \) und~\( b \)
    durch
    \begin{equation*}
        E \paren{a, b} = \set[\Big]{
            \begin{psmallmatrix} x \\ y \end{psmallmatrix} \in \RealNums^2
            \given \frac{x^2}{a^2} + \frac{y^2}{b^2} \leq 1
        }
    \end{equation*}
    Wir sehen sofort dass im Fall~\( a = b \)
    ein Kreis mit Radius~\( a \) und Volumen~\( a^2 \pi \) vorliegt.
    \begin{equation*}
        E \paren{a, a} = \set[\Big]{
            \begin{psmallmatrix} x \\ y \end{psmallmatrix} \in \RealNums^2
            \given \frac{1}{a^2} \paren[\big]{x^2 + y^2} \leq 1
        }
    \end{equation*}

    In jedem Fall k\"onnen wir
    einen Faktor~\( \alpha \coloneqq \frac{b}{a} \) finden
    sodass wir das Volumen bestimmen von
    \begin{equation*}
        E \paren{a, b} = \set[\Big]{
            \begin{psmallmatrix} x \\ y \end{psmallmatrix} \in \RealNums^2
            \given \frac{x^2}{a^2} + \frac{y^2}{b^2} \leq 1
        }
        = \set[\Big]{
            \begin{psmallmatrix} x \\ y \end{psmallmatrix} \in \RealNums^2
            \given \frac{x^2}{a^2} + \frac{y^2}{\paren{\alpha a}^2} \leq 1
        }
        = \set[\Big]{
            \begin{psmallmatrix} x \\ y \end{psmallmatrix} \in \RealNums^2
            \given \frac{1}{a^2}
                \paren[\big]{x^2 + \paren{\tfrac{y}{\alpha}}^2} \leq 1
        }
    \end{equation*}
    Wir erkennen dass
    \begin{gather*}
        \Phi \paren{x}
        = \begin{pmatrix} 1 & 0 \\ 0 & \alpha \end{pmatrix} x
        \eqqcolon A x
        \\
        \begin{aligned}
            \volume[\big]{E \paren{a, b}} &
            = \int_{E \paren{a, b}} 1 \dx
            = \int_{\Phi \paren{E \paren{a, a}}} 1 \dx
            = \int_{E \paren{a, a}} \abs[\big]{\det \TotalD{\Phi}{x}} \dx
            \\ &
            = \abs{\det A} \volume[\big]{E \paren{a, a}}
            = \alpha a^2 \pi
            = a \paren{\alpha a} \pi
            = ab \pi
        \end{aligned}
    \end{gather*}


    \PartEx[Erwartung an 3D]{1}

    Analog zum zweidimensionalen Fall definieren wir zu~\( a, b, c > 0 \)
    einen Ellipsoid~\( E \paren{a, b, c} \subset \RealNums^3 \)
    mit Halbachsen~\( a, b, c \).
    Auch hier sehen wir sofort dass im Fall~\( a = b = c \)
    eine Kugel vorliegt, diesmal mit Volumen~\( \frac{4}{3} a^3 \pi \).

    Ebenso k\"onnen wir Faktoren~\( \alpha \coloneqq \frac{a}{b} \)
    und~\( \beta \coloneqq \frac{a}{c} \) finden
    um den Ellipsoiden in eine Kugel zu transformieren.
    \begin{equation*}
        \Phi \paren{x}
        = \begin{pmatrix}
            1 & 0 & 0 \\ 0 & \alpha & 0 \\ 0 & 0 & \beta
        \end{pmatrix} x
        \eqqcolon A x
    \end{equation*}
    Somit erwarten wir analog zum zweidimensionalen Fall, dass~\(
        \volume[\big]{E \paren{a, b, c}} = \frac{4}{3} abc \pi
    \).
\end{document}
