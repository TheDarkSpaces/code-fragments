\documentclass[../full]{subfiles}


\begin{document}
    \MainEx[Sph\"arische Teilgebiete]{4}
    \label{task:3dint:sphere:parts}

    Es sei~\( \Omega \coloneqq \Omega_1 \cup \Omega_2 \)
    mit~\( \Omega, \Omega_1, \Omega_2 \subset \RealNums^3 \).
    \begin{align*}
        \Omega_1 &= \set[\Big]{
            \begin{psmallmatrix} x \\ y \\ z \end{psmallmatrix} \in \RealNums^3
            \given z \geq 0, \enspace x^2 + y^2 + z^2 \leq 1
        }
        \\
        \Omega_2 &= \set[\Big]{
            \begin{psmallmatrix} x \\ y \\ z \end{psmallmatrix} \in \RealNums^3
            \given -2 \leq z \leq 0, \enspace
                \sqrt{x^2 + y^2} \leq 1 + \frac{z}{2}
        }
    \end{align*}


    \PartEx[Skizze]{1}
    \label{task:3dint:sphere:parts:sketch}

    Wir sehen dass~\( \Omega_1 \) eine Halbkugel beschreibt,
    w\"ahrend~\( \Omega_2 \) einen Kegel darstellt,
    die sich nur in einer Nullmenge \"uberschneiden.
    Das f\"uhrt zur Skizze
    auf \hyperref[task:3dint:sphere:parts:sketch:visual]{%
        Seite~\pageref*{task:3dint:sphere:parts:sketch:visual}%
    }.

    \begin{figure}
        \centering
        \HomeworkPart[Sketch]{Scone}

        \caption*{%
            Skizze zu
            \hyperref[task:3dint:sphere:parts:sketch]{%
                \ref*{task:3dint:sphere:parts}%
                ~\ref*{task:3dint:sphere:parts:sketch}%
            }%
        }
        \label{task:3dint:sphere:parts:sketch:visual}
    \end{figure}


    \PartEx[Volumen und Schwerpunkt]{3}

    Durch eine Integration \"uber Kreisscheiben
    k\"onnen wir leicht die Massen berechnen.
    \begin{align*}
        M_{\Omega_1} &
        = \int_{\Omega_1} 1 \dx[\paren{x, y, z}]
        = \int_{z = 0}^1
            \int_{B_{\sqrt{1 - z^2}} \paren{0}} 1 \dx[\paren{x, y}]
        \dx[z]
        = \int_0^1 \paren{1 - z^2} \pi \dx[z]
        = \paren[\Big]{ 1 - \frac{1}{3} } \pi
        = \frac{2}{3} \pi
        \displaybreak[0] \\
        M_{\Omega_2} &
        = \int_{\Omega_2} 1 \dx[\paren{x, y, z}]
        = \int_{z = -2}^0
            \int_{B_{1 + \frac{z}{2}} \paren{0}} 1 \dx[\paren{x, y}]
        \dx[z]
        = \int_{-2}^0 \paren[\Big]{1 + \frac{z}{2}}^2 \pi \dx[z]
        \\ &
        = \pi \int_{-2}^0 1 + z + \frac{z^2}{4} \dx[z]
        = \paren[\Big]{2 - 2 + \frac{2}{3}} \pi
        = \frac{2}{3} \pi
        \displaybreak[0] \\
        M_\Omega &
        = M_{\Omega_1} + M_{\Omega_2}
        = \frac{2}{3} \pi + \frac{2}{3} \pi
        = \frac{4}{3} \pi
    \end{align*}
    Ebenso k\"onnen wir nun die Schwerpunkte bestimmen.
    \begin{align*}
        S_{\Omega_1} &
        = \frac{1}{M_{\Omega_1}} \int_{\Omega_1}
            \begin{psmallmatrix} x \\ y \\ z \end{psmallmatrix}
        \dx[\paren{x, y, z}]
        \displaybreak[0] \\ & \phantom{{}= \frac{1}{M_{\Omega_1}}}\,
        \begin{aligned}
            \int_{\Omega_1} x \dx[\paren{x, y, z}] &
            = \int_{z = 0}^1
                \int_{B_{\sqrt{1 - z^2}} \paren{0}} x \dx[\paren{x, y}]
            \dx[z]
            = \int_0^1 0 \dx[z]
            = 0
            \\
            \int_{\Omega_1} y \dx[\paren{x, y, z}] &
            = \int_{z = 0}^1
                \int_{B_{\sqrt{1 - z^2}} \paren{0}} y \dx[\paren{x, y}]
            \dx[z]
            = \int_0^1 0 \dx[z]
            = 0
            \\
            \int_{\Omega_1} z \dx[\paren{x, y, z}] &
            = \int_{z = 0}^1
                \int_{B_{\sqrt{1 - z^2}} \paren{0}} z \dx[\paren{x, y}]
            \dx[z]
            = \int_0^1 z \paren{1 - z^2} \pi \dx[z]
            \\ &
            = \paren[\Big]{
                \FunctionBounds{\frac{z^2}{2}}{0}{1}
                - \FunctionBounds{\frac{z^4}{4}}{0}{1}
            } \pi
            = \paren[\Big]{\frac{1}{2} - \frac{1}{4}} \pi
            = \frac{1}{4} \pi
        \end{aligned}
        \displaybreak[0] \\
        S_{\Omega_1} &
        = \frac{3}{2 \pi} \cdot \frac{\pi}{4}
            \begin{psmallmatrix} 0 \\ 0 \\ 1 \end{psmallmatrix}
        = \frac{3}{8} \begin{psmallmatrix} 0 \\ 0 \\ 1 \end{psmallmatrix}
        \displaybreak[1] \\
        S_{\Omega_2} &
        = \frac{1}{M_{\Omega_2}} \int_{\Omega_2}
            \begin{psmallmatrix} x \\ y \\ z \end{psmallmatrix}
        \dx[\paren{x, y, z}]
        \displaybreak[0] \\ & \phantom{{}= \frac{1}{M_{\Omega_2}}}\,
        \begin{aligned}
            \int_{\Omega_2} x \dx[\paren{x, y, z}] &
            = \int_{z = -2}^0
                \int_{B_{1 + \frac{z}{2}} \paren{0}} x \dx[\paren{x, y}]
            \dx[z]
            = \int_{-2}^0 0 \dx[z]
            = 0
            \\
            \int_{\Omega_2} y \dx[\paren{x, y, z}] &
            = \int_{z = -2}^0
                \int_{B_{1 + \frac{z}{2}} \paren{0}} y \dx[\paren{x, y}]
            \dx[z]
            = \int_{-2}^0 0 \dx[z]
            = 0
            \\
            \int_{\Omega_2} z \dx[\paren{x, y, z}] &
            = \int_{z = -2}^0
                \int_{B_{1 + \frac{z}{2}} \paren{0}} z \dx[\paren{x, y}]
            \dx[z]
            = \int_{-2}^0 z \paren[\Big]{1 + \frac{z}{2}}^2 \pi \dx[z]
            \\ &
            = \pi \int_{-2}^0 z + z^2 + \frac{z^3}{4} \dx[z]
            = \paren[\Big]{-2 + \frac{8}{3} - 1} \pi
            = -\frac{1}{3} \pi
        \end{aligned}
        \displaybreak[0] \\
        S_{\Omega_2} &
        = \frac{3}{2 \pi} \cdot \frac{-\pi}{3}
            \begin{psmallmatrix} 0 \\ 0 \\ 1 \end{psmallmatrix}
        = -\frac{1}{2} \begin{psmallmatrix} 0 \\ 0 \\ 1 \end{psmallmatrix}
        \displaybreak[1] \\
        S_\Omega &
        = \frac{1}{M_\Omega}
            \paren[\big]{M_{\Omega_1} S_{\Omega_1} + M_{\Omega_2} S_{\Omega_2}}
        = \frac{3}{4 \pi}
            \paren[\Big]{
                \frac{3}{8} \cdot \frac{2}{3} \pi
                - \frac{1}{2} \cdot \frac{2}{3} \pi
            }
            \begin{psmallmatrix} 0 \\ 0 \\ 1 \end{psmallmatrix}
        \\ &
        = \frac{1}{2} \paren[\Big]{\frac{3}{8} - \frac{1}{2}}
            \begin{psmallmatrix} 0 \\ 0 \\ 1 \end{psmallmatrix}
        = -\frac{1}{16} \begin{psmallmatrix} 0 \\ 0 \\ 1 \end{psmallmatrix}
    \end{align*}
\end{document}
