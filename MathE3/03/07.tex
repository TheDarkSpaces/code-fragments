\documentclass[../full]{subfiles}


\begin{document}
    \MainEx[Grenzwerte Uneigentlicher Integrale]{2}

    Zu \( p \in \RealNums^2 \)~beliebig und~\( R > 0 \) bezeichne~\(
        B_R \paren{p} = \set[\big]{x \in \RealNums^2 \given \norm{x - p} \leq R}
    \) die Kreisscheibe um~\( p \) mit Radius~\( R \).
    Zu berechnen ist
    \begin{equation*}
        I_R \coloneqq \int_{B_R \paren{p}} \eEuler^{-\norm{x - p}^2} \dx
    \end{equation*}
    unter Verwendung der Euklidischen Norm~\(
        \norm{{}\cdot{}} \coloneqq \norm{{}\cdot{}}_2
    \).

    Wir beobachten f\"ur eine Translation~\(
        T_p \paren{x} \coloneqq x + p,
    \) um den Punkt~\( p \) dass~\(
        B_R \paren{p} = T_p \paren[\big]{B_R \paren{0}}
    \) und somit
    \begin{equation*}
        \int_{B_R \paren{p}} \eEuler^{-\norm{x - p}^2} \dx
        = \int_{B_R \paren{0}}
            \eEuler^{-\norm{x}^2} \abs[\big]{\TotalD{T_p}{x}}
        \dx
        = \int_{B_R \paren{0}} \eEuler^{-\norm{x}^2} \dx
    \end{equation*}
    Damit verbleiben wir mit einer Kreisscheibe um den Ursprung,
    die wir leicht durch Polarkoordinaten ausdr\"ucken k\"onnen:
    \begin{gather*}
        \Phi \colon
            \RealNumsZeroPos \times \IntervalCO{0}{2 \pi} \to \RealNums^2,
            \enspace
            \begin{pmatrix} r \\ \varphi \end{pmatrix}
                \mapsto r \begin{pmatrix}
                    \cos \varphi \\ \sin \varphi
                \end{pmatrix}
        \\
        \TotalD{\Phi}{r, \varphi} = \begin{pmatrix}
            \cos \varphi & -r \sin \varphi \\
            \sin \varphi & \phantom{-}r \cos \varphi
        \end{pmatrix},
        \quad
        \det \TotalD{\Phi}{r, \varphi}
        = r \paren{\cos^2 \varphi + \sin^2 \varphi} = r
        \displaybreak[1] \\
        B_R \paren{0} = \set[\big]{x \in \RealNums^2 \given \norm{x} \leq R}
        = \Phi \paren[\Big]{
            \set[\Bigg]{
                \begin{pmatrix} r \\ \varphi \end{pmatrix}
                    \in \RealNumsZeroPos \times \IntervalCO{0}{2 \pi}
                \given r \leq R
            }
        }
        \eqqcolon \Omega_R
        \displaybreak[0] \\
        \begin{aligned}
            I_R &
            = \int_{B_R \paren{p}} \eEuler^{-\norm{x - p}^2} \dx
            = \int_{B_R \paren{0}} \eEuler^{-\norm{x}^2} \dx
            = \int_{\Omega_R}
                \eEuler^{-\norm{\Phi \paren{r, \varphi}}^2}
                \abs[\big]{\det \TotalD{\Phi}{r, \varphi}}
            \dx[\paren{r, \varphi}]
            \\ &
            = \int_{\Omega_R}
                \eEuler^{-r^2 \paren{\cos^2 \varphi + \sin^2 \varphi}} \abs{r}
            \dx[\paren{r, \varphi}]
            = \int_{\Omega_R} r \eEuler^{-r^2} \dx[\paren{r, \varphi}]
            = \pi \int_R^0 -2r \eEuler^{-r^2} \dx[r]
            \\ &
            = \pi \int_{-R^2}^0 \eEuler^{u} \dx[u]
            = \pi \FunctionBounds{\eEuler^u}{-R^2}{0}
            = \paren[\Big]{1 - \frac{1}{\eEuler^{R^2}}} \pi
        \end{aligned}
        \displaybreak[1] \\
        \int_{\RealNums^2} \eEuler^{-\norm{x - p}^2} \dx
        = \lim_{R \to \infty} I_R
        = \lim_{R \to \infty} \paren[\Big]{1 - \frac{1}{\eEuler^{R^2}}} \pi
        = \paren{1 - 0} \pi
        = \pi
    \end{gather*}
\end{document}
