\documentclass[../full]{subfiles}


\begin{document}
    \MainEx[Masse und Schwerpunkt von Mengen]{4}


    \PartEx[Zusammengesetzten Mengen]{2}

    Es seien \( \Omega_1, \Omega_2 \in \RealNums^n \)~disjunkt und messbar,
    mit Massen~\( M_{\Omega_1}, M_{\Omega_2} \)
    und Schwerpunkten~\( S_{\Omega_1}, S_{\Omega_2} \).
    Dann ergibt sich f\"ur eine zusammengesetzte Menge~\(
        \Omega \coloneqq \Omega_1 \cup \Omega_2
    \) als Masse und Schwerpunkt
    \begin{align*}
        M_\Omega &
        = \int_\Omega \rho \paren{x} \dx
        = \int_{\Omega_1 \cup \Omega_2} \rho \paren{x} \dx
        = \int_{\Omega_1} \rho \paren{x} \dx
            + \int_{\Omega_2} \rho \paren{x} \dx
        = M_{\Omega_1} + M_{\Omega_2}
        \\
        S_\Omega &
        = \begin{aligned}[t]
            \frac{1}{M_\Omega} \int_\Omega x \rho \paren{x} \dx &
            = \frac{1}{M_{\Omega_1} + M_{\Omega_2}}
            \paren[\Big]{
            \int_{\Omega_1} x \rho \paren{x} \dx
            + \int_{\Omega_2} x \rho \paren{x} \dx
            }
            \\ &
            = \frac{1}{M_{\Omega_1} + M_{\Omega_2}}
            \paren[\Big]{
            M_{\Omega_1}
            \frac{1}{M_{\Omega_1}} \int_{\Omega_1} x \rho \paren{x} \dx
            + M_{\Omega_2}
            \frac{1}{M_{\Omega_2}} \int_{\Omega_2} x \rho \paren{x} \dx
            }
            \\ &
            = \frac{1}{M_{\Omega_1}
                + M_{\Omega_2}} \paren[\big]{
                    M_{\Omega_1} S_{\Omega_1} + M_{\Omega_2} S_{\Omega_2}
                }
        \end{aligned}
    \end{align*}


    \PartEx[Untermengen]{2}

    Zur Bestimmung der Fl\"ache~\( M \)
    und des geometrischen Schwerpunkts~\( S \) von Mengen
    setzen wir die Dichtefunktion~\( \rho \colon x \mapsto 1 \) an.

    Zu~\( R, a > 0 \) mit~\( 2 a^2 \leq R^2 \) definieren wir die Mengen
    \begin{equation*}
        D = \set[\big]{ x \in \RealNums^2 \given \norm{x}_2 \leq R },
        \qquad
        Q = \IntervalClosed{-a}{a} \times \IntervalClosed{0}{a},
        \qquad
        X = D \setminus Q
    \end{equation*}
    F\"ur~\( D \) und~\( Q \) sind Fl\"ache und gemetrischer Schwerpunkt
    sofort ersichtlich als
    \begin{equation*}
        \begin{aligned}
            M_D &= \pi R^2
            \\
            S_D &= \begin{pmatrix} 0 \\ 0 \end{pmatrix}
        \end{aligned}
        \qquad \text{und} \qquad
        \begin{aligned}
            M_Q &= 2 a^2
            \\
            S_Q &= \frac{1}{2} \begin{pmatrix} 0 \\ a \end{pmatrix}
        \end{aligned}
    \end{equation*}

    Wir sehen, dass~\( Q \subset D \)
    weil~\( \max_{x \in Q} \norm{x}_2 = a \sqrt{2} \leq R \).
    Das erlaubt uns, \( D = Q \cup X \) zu setzen,
    wobei \( Q \)~und~\( X \) disjunkt sind.
    Damit gilt f\"ur Fl\"ache und Schwerpunkt von~\( X \)
    \begin{gather*}
        M_X = M_D - M_Q
        = \pi R^2 - 2 a^2
        \\
        S_X = \frac{1}{M_X} \paren[\big]{M_D S_D - M_Q S_Q}
        = \frac{1}{2 a^2 - \pi R^2} \begin{pmatrix} 0 \\ a^3 \end{pmatrix}
    \end{gather*}
    Im Fall~\( a = \frac{R}{2} \) ergibt sich dann
    \begin{gather*}
        \begin{aligned}
            M_Q &= \frac{R^2}{2}
            \\
            S_Q &= \frac{1}{4} \begin{pmatrix} 0 \\ R \end{pmatrix}
        \end{aligned}
        \qquad \text{und} \qquad
        \begin{aligned}
        M_X &= \paren[\Big]{\pi - \frac{1}{2}} R^2
        \\
        S_X &= \frac{1}{4 \paren{1 - 2 \pi}}
            \begin{pmatrix} 0 \\ R \end{pmatrix}
        \end{aligned}
    \end{gather*}
\end{document}
