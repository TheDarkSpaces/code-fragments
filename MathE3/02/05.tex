\documentclass[../full]{subfiles}


\begin{document}
    \MainEx[Integration auf Allgemeineren Gebieten]{4}
    \label{task:3dint:generic}

    Es sei~\( D \subset \RealNums^3 \)
    die Menge die begrenzt wird durch
    \begin{align*}
        E_1 &= \set[\Big]{
            \begin{psmallmatrix} x \\ y \\ z \end{psmallmatrix} \in \RealNums^3
            \given x = 0
        }
        \\
        E_2 &= \set[\Big]{
            \begin{psmallmatrix} x \\ y \\ z \end{psmallmatrix} \in \RealNums^3
            \given y = 0
        }
        \\
        E_3 &= \set[\Big]{
            \begin{psmallmatrix} x \\ y \\ z \end{psmallmatrix} \in \RealNums^3
            \given z = 0
        }
        \\
        E_4 &= \set[\Big]{
            \begin{psmallmatrix} x \\ y \\ z \end{psmallmatrix} \in \RealNums^3
            \given x - y - z = 0
        }
        \\
        E_5 &= \set[\Big]{
            \begin{psmallmatrix} x \\ y \\ z \end{psmallmatrix} \in \RealNums^3
            \given y + z = 1
        }
    \end{align*}



    \PartEx[Skizze]{2}
    \label{task:3dint:generic:sketch}

    Wir bezeichnen den Schnittpunkt der drei Mengen~\( E_i, E_j, E_k \)
    mit~\( 1 \leq i < j < k \leq 5 \) als~\( p_{ijk} \).
    Dies geschieht ohne Einschr\"ankung der Allgemeinheit
    weil mit der Kommutativit\"at gilt dass~\( E_i \cap E_j = E_j \cap E_i \).

    Wir stellen fest dass~\( \forall k \leq 4 \colon p_{ijk} = 0 \).
    Weiterhin existieren weder~\( p_{145} \) noch~\( p_{235} \).
    \begin{equation*}
        p_0 \coloneqq
        p_{ijk} = \begin{pmatrix} 0 \\ 0 \\ 0 \end{pmatrix},
        \; k \leq 4,
        \qquad
        p_{125} = \begin{pmatrix} 0 \\ 0 \\ 1 \end{pmatrix},
        \qquad
        p_{135} = \begin{pmatrix} 0 \\ 1 \\ 0 \end{pmatrix},
        \qquad
        p_{245} = \begin{pmatrix} 1 \\ 0 \\ 1 \end{pmatrix},
        \qquad
        p_{345} = \begin{pmatrix} 1 \\ 1 \\ 0 \end{pmatrix}
    \end{equation*}
    Mit diesen Punkten k\"onnen wir die Skizze von~\( D \)
    auf \hyperref[task:3dint:generic:sketch:visual]{%
        Seite~\pageref*{task:3dint:generic:sketch:visual}%
    } anfertigen.

    \begin{figure}
        \centering
        \HomeworkPart[Sketch]{Tetraeder}

        \caption*{%
            Skizze zu
            \hyperref[task:3dint:generic:sketch]{%
                \ref*{task:3dint:generic}~\ref*{task:3dint:generic:sketch}%
            }%
        }
        \label{task:3dint:generic:sketch:visual}
    \end{figure}


    \PartEx[Integration mit Fubini]{2}

    Wir erkennen dass wir~\( D \) schreiben k\"onnen als
    \begin{equation*}
        D = \set[\Big]{
            \begin{psmallmatrix} x \\y \\z \end{psmallmatrix} \in \RealNums^3
            \given x, y, z \geq 0 \enspace \land \enspace x \leq y + z \leq 1
        }
        = \set[\Big]{
            \begin{psmallmatrix} x \\y \\z \end{psmallmatrix} \in \RealNums^3
            \given z \in \IntervalClosed{0}{1}, \;
                y \in \IntervalClosed{0}{1 - z}, \;
                x \in \IntervalClosed{0}{z + y}
        }
    \end{equation*}
    Damit k\"onnen wir recht einfach integrieren und erhalten
    \begin{align*}
        \int_D x \dx[\paren{x, y, z}] &
        = \int_{z = 0}^1
            \int_{y = 0}^{1 - z}
                \int_{x = 0}^{z + y} x \dx
            \dx[y]
        \dx[z]
        \\ &
        = \int_{z = 0}^1
            \int_{y = 0}^{1 - z} \frac{\paren{z + y}^2}{2} \dx[y]
        \dx[z]
        = \int_{z = 0}^1
            \int_{y = 0}^{1 - z} \frac{z^2}{2} + \frac{y^2}{2} + zy \dx[y]
        \dx[z]
        \\ &
        = \int_{z = 0}^1
            \int_{y = 0}^{1 - z} zy \dx[y]
            + \int_{y = 0}^{1 - z} \frac{z^2}{2} \dx[y]
            + \int_{y = 0}^{1 - z} \frac{y^2}{2} \dx[y]
        \dx[z]
        \\ &
        = \int_0^1
            \frac{\paren{1 - z} z^2}{2}
            + \frac{\paren{1 - z}^2 z}{2}
            + \frac{\paren{1 - z}^3}{6}
        \dx[z]
        \\ &
        = \int_0^1
            \frac{ \paren{1 - z} \paren{3 z^2 + 3z - 3 z^2 + 1 - 2z + z^2} }{6}
        \dx[z]
        \\ &
        = \frac{1}{6} \int_0^1 1 - z^3 \dx[z]
        = \frac{1}{6} \paren[\Big]{1 - \frac{1}{4}}
        = \frac{3}{24}
        = \frac{1}{8}
    \end{align*}
\end{document}
