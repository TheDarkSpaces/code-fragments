\documentclass[../full]{subfiles}


\newcommand\DecimalPeriod[1]{\overline{#1}}

\newcommand\ErrorVar[1][]{\varepsilon_{#1}}

\begin{document}
    \MainEx[Mehrdimensionale Riemann-Summen]{4}

    \begin{gather*}
        f \paren{x, y} = x \paren{1 - x} y
        \\
        I = \int_{\IntervalClosed{0}{1}^2} f \paren{x, y} \dx[\paren{x, y}]
    \end{gather*}


    \PartEx[Approximation durch Riemann-Summen]{3}

    Wir sehen sofort, dass~\( f_2 \paren{y} = y \) streng monoton wachsend ist
    mit~\( f_2 \paren{\RealNumsZeroPos} = \RealNumsZeroPos \),
    w\"ahrend f\"ur~\( f_1 \paren{x} = x \paren{1 - x} \)
    offensichtlich gilt dass~\(
        f_1 \paren{0} = f_1 \paren{1} = 0 < f_1 \paren[\big]{\widetilde{x}}
    \) mit~\( \widetilde{x} \in \IntervalOpen{0}{1} \).
    Weil~\( f_1' \paren{x} = 1 - 2x \) nur bei~\( x = \frac{1}{2} \)
    eine Nullstelle besitzt,
    und weil~\( f_1'' \paren{x} = -2 < 0 \),
    muss dieser Punkt eine Maximalstelle sein.
    Damit ist~\( f_1 \) streng monoton wachsend
    auf~\( \IntervalClosed[\big]{0}{\frac{1}{2}} \),
    und streng monoton fallend auf~\( \IntervalClosed[\big]{\frac{1}{2}}{1} \).
    Wir sehen au\ss erdem, dass
    \begin{equation*}
        f_1 \paren[\big]{\frac{1}{2} + x}
        = \paren[\big]{\frac{1}{2} + x}
            \paren[\Big]{1 - \paren[\big]{\frac{1}{2} + x}}
        = \paren[\big]{\frac{1}{2} + x} \paren[\big]{\frac{1}{2} - x}
        = \paren[\big]{\frac{1}{2} - x} \paren[\big]{\frac{1}{2} + x}
        = \paren[\big]{\frac{1}{2} - x}
            \paren[\Big]{1 - \paren[\big]{\frac{1}{2} - x}}
        = f_1 \paren[\big]{\frac{1}{2} - x}
    \end{equation*}

    Es ergeben sich f\"ur eine Zerlegung
    in \( x \)-Richtung mit Feinheit~\( h_x = \frac{1}{3} \)
    und in \( y \)-Richtung mit Feinheit~\( h_y = \frac{1}{2} \)
    die Intervalle
    \begin{gather*}
        X_1 = \IntervalClosed[\Big]{0}{\frac{1}{3}}, \qquad
        X_2 = \IntervalClosed[\Big]{\frac{1}{3}}{\frac{2}{3}}, \qquad
        X_3 = \IntervalClosed[\Big]{\frac{2}{3}}{1}
        \\
        Y_1 = \IntervalClosed[\Big]{0}{\frac{1}{2}}, \qquad
        Y_2 = \IntervalClosed[\Big]{\frac{1}{2}}{1}
    \end{gather*}
    mit~\( R_{ij} \coloneqq X_i \times Y_j \)
    und~\( \volume{R_{ij}} = h_x \cdot h_y = \frac{1}{6} \).

    Aus diesen Eigenschaften l\"asst sich zum Beispiel\footnote{
        Es lassen sich auch andere Koordinaten~\( \paren{x, y}_{\min} \) finden,
        die diese Eigenschaften erf\"ullen,
        wie etwa \mbox{\( \paren{x, y}_{\min} = \paren{0, \frac{1}{2}} \)}
        oder auch \mbox{\( \paren{x, y}_{\min} = \paren{\frac{1}{3}, 0} \)}
        f\"ur~\( R_{11} \).
    } \autoref{task:Riemann:2DSum:table} zusammenstellen:
    \begin{table}
        \centering
        \caption{Eine m\"ogliche Wertetabelle}
        \label{task:Riemann:2DSum:table}

        \newcommand\xyArrange[3][]{ \paren{#2, #3}_{#1} }
        \begin{tabular}{ l || *5{c|} c || c }
                & \( R_{11} \) & \( R_{21} \) & \( R_{31} \)
                & \( R_{12} \) & \( R_{22} \) & \( R_{32} \)
                & Riemann-Summe
            \\ \hline \hline
            \( \phantom{f} \xyArrange[\min]{x}{y} \)
                & \( \xyArrange{0}{0} \)
                & \( \xyArrange{\frac{1}{3}}{0} \)
                & \( \xyArrange{1}{0} \)
                & \( \xyArrange{0}{\frac{1}{2}} \)
                & \( \xyArrange{\frac{2}{3}}{\frac{1}{2}} \)
                & \( \xyArrange{1}{\frac{1}{2}} \)
            \\
            \( f \xyArrange[\min]{x}{y} \)
                & \( 0 \) & \( 0 \) & \( 0 \)
                & \( 0 \) & \( \frac{1}{9} \) & \( 0 \)
                & \( U_R = \frac{1}{6} \cdot \frac{1}{9} = \frac{1}{54} \)
            \\ \hline
            \( \phantom{f} \xyArrange[\max]{x}{y} \)
                & \( \xyArrange{\frac{1}{3}}{\frac{1}{2}} \)
                & \( \xyArrange{\frac{1}{2}}{\frac{1}{2}} \)
                & \( \xyArrange{\frac{2}{3}}{\frac{1}{2}} \)
                & \( \xyArrange{\frac{1}{3}}{1} \)
                & \( \xyArrange{\frac{1}{2}}{1} \)
                & \( \xyArrange{\frac{2}{3}}{1} \)
            \\
            \( f \xyArrange[\max]{x}{y} \)
                & \( \frac{1}{9} \) & \( \frac{1}{8} \) & \( \frac{1}{9} \)
                & \( \frac{2}{9} \) & \( \frac{1}{4} \) & \( \frac{2}{9} \)
                & \( O_R = \frac{1}{6} \cdot \frac{25}{24} = \frac{25}{144} \)
            \\ \hline
            \( \phantom{f} \xyArrange[\mathrm{Mitte}]{x}{y} \)
                & \( \xyArrange{\frac{1}{6}}{\frac{1}{4}} \)
                & \( \xyArrange{\frac{1}{2}}{\frac{1}{4}} \)
                & \( \xyArrange{\frac{5}{6}}{\frac{1}{4}} \)
                & \( \xyArrange{\frac{1}{6}}{\frac{3}{4}} \)
                & \( \xyArrange{\frac{1}{2}}{\frac{3}{4}} \)
                & \( \xyArrange{\frac{5}{6}}{\frac{3}{4}} \)
            \\
            \( f \xyArrange[\mathrm{Mitte}]{x}{y} \)
                & \( \frac{5}{144} \) & \( \frac{1}{16} \) & \( \frac{5}{144} \)
                & \( \frac{5}{48} \) & \( \frac{3}{16} \) & \( \frac{5}{48} \)
                & \( M_R = \frac{1}{6} \cdot \frac{19}{36} = \frac{19}{216} \)
        \end{tabular}
    \end{table}


    \PartEx[Exakter Wert und Approximationsfehler]{1}

    % many thanks to the calculator at
    % http://www.calcul.com/show/calculator/fraction-to-recurring-decimal
    \begin{gather*}
        I = \int_{\IntervalClosed{0}{1}^2}
                \paren{x - x^2} \cdot y \dx[\paren{x, y}]
        = \int_0^1 x - x^2 \dx \cdot \int_0^1 y \dx[y]
        = \FunctionBounds{\frac{x^2}{2} - \frac{x^3}{3}}{0}{1}
            \cdot \FunctionBounds{\frac{y^2}{2}}{0}{1}
        = \frac{1}{6} \cdot \frac{1}{2}
        = \frac{1}{12}
        \\
        \begin{aligned}
            \ErrorVar[U] \coloneqq \abs{I - U_R} &
            = \frac{1}{12} - \frac{1}{54} = \frac{7}{108}
            = 0.06 \DecimalPeriod{481}
            \\
            \ErrorVar[O] \coloneqq \abs{I - O_R} &
            = \frac{25}{144} - \frac{1}{12} = \frac{13}{144}
            = 0.0902 \DecimalPeriod{7}
            \\
            \ErrorVar[M] \coloneqq \abs{I - M_R} &
            = \frac{19}{216} - \frac{1}{12} = \frac{1}{216}
            = 0.004 \DecimalPeriod{629}
        \end{aligned}
        \\
        \ErrorVar[M] \ll \ErrorVar[U] < \ErrorVar[O]
    \end{gather*}
\end{document}
