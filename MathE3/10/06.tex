\documentclass[../full]{subfiles}


\begin{document}
    \MainEx[\( \ComplexNums \)omplexe Abbildungen]{4}

    \begin{equation*}
        f \colon \ComplexNums \to \ComplexNums, \enspace z \mapsto  \eEuler^z
    \end{equation*}


    \PartEx[Sinus und Cosinus]{1}

    \begin{gather*}
        \cos \paren{z}
        = \frac{\eEuler^{\iComplex z} + \eEuler^{-\iComplex z}}{2}
        , \qquad
        \sin \paren{z}
        = \frac{\eEuler^{\iComplex z} - \eEuler^{-\iComplex z}}{2 \iComplex}
        \displaybreak[0] \\
        \cos z + \iComplex \sin z
        = \frac{\eEuler^{\iComplex z} + \eEuler^{-\iComplex z}}{2}
            + \iComplex \frac
                {\eEuler^{\iComplex z} - \eEuler^{-\iComplex z}}
                {2 \iComplex}
        = \frac{\eEuler^{\iComplex z} + \eEuler^{-\iComplex z}}{2}
            + \frac{\eEuler^{\iComplex z} - \eEuler^{-\iComplex z}}{2}
        = \frac{2 \eEuler^{\iComplex z}}{2}
        = \eEuler^{\iComplex z}
    \end{gather*}


    \PartEx[\( \ComplexNums \)omplexer Pythagoras]{1}

    \begin{equation*}
        \cos^2 z + \sin^2 z
        = \frac
                {\paren{\eEuler^{\iComplex z} + \eEuler^{-\iComplex z}}^2}
                {4}
            + \frac
                {\paren{\eEuler^{\iComplex z} - \eEuler^{-\iComplex z}}^2}
                {4 \iComplex^2}
        = \frac{\eEuler^{2 \iComplex z} + \eEuler^{-2 \iComplex z} + 2}{4}
            - \frac{\eEuler^{2 \iComplex z} + \eEuler^{-2 \iComplex z} - 2}{4}
        = \frac{4}{4}
        = 1
    \end{equation*}


    \PartEx[Ableitungen]{1}

    \begin{gather*}
        \paren[\big]{\eEuler^z}' = \eEuler^z
        \displaybreak[0] \\
        \sin' z
        = \paren[\Big]{\frac
            {\eEuler^{\iComplex z} - \eEuler^{-\iComplex z}}
            {2 \iComplex}
        }'
        = \frac
            {\iComplex \eEuler^{\iComplex z}
                + \iComplex \eEuler^{-\iComplex z}}
            {2 \iComplex}
        = \frac{\eEuler^{\iComplex z} + \eEuler^{-\iComplex z}}{2}
        = \cos z
        \displaybreak[0] \\
        \cos' z
        = \paren[\Big]{\frac
            {\eEuler^{\iComplex z} + \eEuler^{-\iComplex z}}
            {2}
        }'
        = \frac
            {\iComplex \eEuler^{\iComplex z}
                - \iComplex \eEuler^{-\iComplex z}}
            {2}
        = \frac
            {\iComplex^2 \eEuler^{\iComplex z}
                - \iComplex^2 \eEuler^{-\iComplex z}}
            {2 \iComplex}
        = -\frac{\eEuler^{\iComplex z} - \eEuler^{-\iComplex z}}{2 \iComplex}
        = -\sin z
    \end{gather*}


    \PartEx[Isomorphie]{1}

    \begin{gather*}
        \eEuler^z
        = f \paren{z}
        \coloneqq f \paren{x + y \iComplex}
        = \eEuler^{x + y \iComplex}
        = \eEuler^{x} \eEuler^{y \iComplex}
        = \eEuler^x \cos y + \eEuler^x \iComplex \sin y
        \coloneqq u \paren{x, y} + v \paren{x, y} \iComplex
        \displaybreak[0] \\
        F \colon \RealNums^2 \to \RealNums^2
        , \enspace
        \begin{pmatrix} x \\ y \end{pmatrix}
        \mapsto \begin{pmatrix} u \paren{x, y} \\ v \paren{x, y} \end{pmatrix}
        = \begin{pmatrix} \eEuler^x \cos y \\ \eEuler^x \sin y \end{pmatrix}
        = \eEuler^x \begin{pmatrix} \cos y \\ \sin y \end{pmatrix}
        \displaybreak[0] \\
        \TotalD{F}{x, y}
        = \begin{pmatrix}
            \eEuler^x \cos y & -\eEuler^x \sin y \\
            \eEuler^x \sin y & \phantom{-}\eEuler^x \cos y
        \end{pmatrix}
        = \eEuler^x \begin{pmatrix}
            \cos y & -\sin y \\ \sin y & \phantom{-}\cos y
        \end{pmatrix}
        , \quad
        f' \paren{z}
        = f' \paren{x + y \iComplex}
        = \eEuler^x \eEuler^{y \iComplex}
    \end{gather*}
    Wir sehen dass~\( \TotalD{F}{x, y} \)
    einer Skalierung um~\( \eEuler^x \) entspricht,
    kombiniert mit einer Rotation um den Winkel~\( y \).
    Selbiges geht aus der Betrachtung von~\( f' \paren{x + y \iComplex} \)
    in der \( \ComplexNums \)omplexen Ebene hervor.
\end{document}
