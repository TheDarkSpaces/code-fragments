\documentclass[../full]{subfiles}


\begin{document}
    \MainEx[Einheitsnormalen der Ebene]{2}

    Die Ebene~\( E \subset \RealNums^4 \) sei aufgespannt von den Vektoren
    \begin{equation*}
        x = \begin{pmatrix} 1 \\ 0 \\ 1 \\ 1 \end{pmatrix}
        , \quad
        y = \begin{pmatrix}
            \phantom{-}1 \\ -1 \\ \phantom{-}0 \\ -1
        \end{pmatrix}
        , \quad
        z = \begin{pmatrix} 1 \\ 1 \\ 1 \\ 1 \end{pmatrix}
    \end{equation*}
    Wir suchen die beiden Einheitsnormalenvektoren zu~\( E \).


    \PartEx[Gleichungssystem]{1}

    Wir wissen, dass f\"ur einen Normalenvektor~\( n \in \RealNums^4 \) gilt
    \begin{equation*}
        \InnerProd{x}{n} = \InnerProd{y}{n} = \InnerProd{z}{n} = 0
    \end{equation*}
    Wenn wir das in ein Lineares Gleichungssystem einsetzen, erhalten wir
    \begin{gather*}
        \begin{pmatrix}
            1 & \phantom{-}0 & 1 & \phantom{-}1 \\
            1 & -1 & 0 & -1 \\
            1 & \phantom{-}1 & 1 & \phantom{-}1
        \end{pmatrix}
        \to
        \begin{pmatrix}
            1 & 0 & 1 & 1 \\
            0 & 1 & 0 & 0 \\
            0 & 1 & 1 & 2
        \end{pmatrix}
        \to
        \begin{pmatrix}
            1 & 0 & 1 & 1 \\
            0 & 1 & 0 & 0 \\
            0 & 0 & 1 & 2
        \end{pmatrix}
        \to
        \begin{pmatrix}
            1 & 0 & 0 & -1 \\
            0 & 1 & 0 & \phantom{-}0 \\
            0 & 0 & 1 & \phantom{-}2
        \end{pmatrix}
        , \quad
        n = \lambda \begin{pmatrix}
            \phantom{-}1 \\ \phantom{-}0 \\ -2 \\ \phantom{-}1
        \end{pmatrix}
        \eqqcolon \lambda n_*
        , \enspace
        \lambda \in \RealNums
    \end{gather*}

    Weiterhin suchen wir nur die Einheitsnormalen,
    also muss gelten~\( \norm{n} = \norm{n}^2 = 1 \).
    \begin{equation*}
        1 = \norm{n}^2 = \lambda^2 \norm{n_*}^2 = 6 \lambda^2
        \quad \Rightarrow \quad
        \lambda = \frac{\sqrt{6}}{6}
        , \enspace
        n = \pm \frac{\sqrt{6}}{6} n_*
    \end{equation*}


    \PartEx[\"Au\ss eres Produkt]{1}

    Wir erhalten einen Normalenvektor aus dem \"Au\ss eren Produkt
    \begin{gather*}
        n_*
        \coloneqq \OuterProd{x, y, z}
        = -\StandardVect{1} \begin{vmatrix}
            0 & -1 & 1 \\
            1 &  0 & 1 \\
            1 & -1 & 1
        \end{vmatrix} % = -1
        + \StandardVect{2} \begin{vmatrix}
            1 &  1 & 1 \\
            1 &  0 & 1 \\
            1 & -1 & 1
        \end{vmatrix} % = 0
        - \StandardVect{3} \begin{vmatrix}
            1 &  1 & 1 \\
            0 & -1 & 1 \\
            1 & -1 & 1
        \end{vmatrix} % = 2
        + \StandardVect{4} \begin{vmatrix}
            1 &  1 & 1 \\
            0 & -1 & 1 \\
            1 &  0 & 1
        \end{vmatrix} % = 1
        = \begin{pmatrix}
            \phantom{-}1 \\ \phantom{-}0 \\ -2 \\ \phantom{-}1
        \end{pmatrix}
    \end{gather*}
    Damit ist der Normaleneinheitsvektor
    \begin{equation*}
        n = \pm \frac{\OuterProd{x, y, z}}{\norm[\big]{\OuterProd{x, y, z}}}
        = \pm \frac{1}{\norm{n_*}} n_*
        = \pm \frac{1}{\sqrt{6}} n_*
        = \pm \frac{\sqrt{6}}{6} n_*
    \end{equation*}
\end{document}
