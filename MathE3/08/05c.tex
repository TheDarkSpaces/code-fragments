\documentclass[../full]{subfiles}


\begin{document}
    \MainEx[Rotation Linearer Vektorfelder]{4}

    Wir betrachten die Rotation von Vektorfeldern~\(
        f \colon \RealNums^3 \to \RealNums^3
    \).
    Die Definition der Rotation in~\( \RealNums^3 \) lautet
    \begin{equation*}
        \curl f = \begin{pmatrix}
            \frac{\partial f_3}{\partial x_2}
                - \frac{\partial f_2}{\partial x_3} \\
            \frac{\partial f_1}{\partial x_3}
                - \frac{\partial f_3}{\partial x_1} \\
            \frac{\partial f_2}{\partial x_1}
                - \frac{\partial f_1}{\partial x_2}
        \end{pmatrix}
    \end{equation*}


    \PartEx[Identit\"at]{1}

    \begin{equation*}
        f_1 \paren{x} = x
        , \quad
        \curl f_1 = 0 \in \RealNums^3
    \end{equation*}


    \PartEx[]{1}

    \begin{equation*}
        f_2 \paren{x} = \begin{pmatrix}
            -x_2 \\ \phantom{-} x_1 \\ \phantom{-} 0
        \end{pmatrix}
        , \quad
        \curl f_2 = \begin{pmatrix} 0 \\ 0 \\ 2 \end{pmatrix}
    \end{equation*}


    \PartEx[]{1}

    \begin{equation*}
        f_3 \paren{x} = \begin{pmatrix}
            2 x_1 - 3 x_2 \\ 4 x_1 - x_2 - x_3 \\ 2 x_1 + x_3
        \end{pmatrix}
        , \quad
        \curl f_3 = \begin{pmatrix}
            \phantom{-} 1 \\ -2 \\ \phantom{-} 7
        \end{pmatrix}
    \end{equation*}


    \NoPartExScore[Allgemeiner Fall]{1}

    F\"ur eine beliebige Matrix~\( A \in \RealNums^{3 \times 3} \) gilt
    \begin{equation*}
        f_A \paren{x} = Ax
        = \begin{pmatrix}
            A_{11} x_1 + A_{12} x_2 + A_{13} x_3 \\
            A_{21} x_1 + A_{22} x_2 + A_{23} x_3 \\
            A_{31} x_1 + A_{32} x_2 + A_{33} x_3
        \end{pmatrix}
        , \quad
        \curl f_A = \begin{pmatrix}
            A_{32} - A_{23} \\ A_{13} - A_{31} \\ A_{21} - A_{12}
        \end{pmatrix}
    \end{equation*}
\end{document}
