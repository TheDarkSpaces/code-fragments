\documentclass[../full]{subfiles}


\newcommand\NormalFunc{
    \vv{n}
}

\begin{document}
    \MainEx[Geschwindigkeitsfeld und Fluss]{2}

    Zu einem gegebenen Geschwindigkeitsfeld
    \begin{equation*}
        V \colon \RealNums^3 \to \RealNums^3
        , \enspace
        \begin{pmatrix} x \\ y \\ z \end{pmatrix}
            \mapsto \begin{pmatrix} x^2 \\ xy \\ yz \end{pmatrix}
    \end{equation*}
    suchen wir den Fluss~\( S = \int_{\partial \Omega} V \vecdF \)
    durch die Oberfl\"ache der Einheitskugel~\( \Omega = B_0 \paren{1} \)
    in Richtung der \"au\ss eren Normalen.

    Weil alle Werte von~\( \partial \Omega \)
    auf der Niveaufl\"ache~\( g \paren{x} \coloneqq \norm{x}^2 = 1 \) liegen,
    wissen wir dass~\( \nabla g \paren{x} = 2x \)
    eine Normale zu~\( x \in \partial \Omega \) darstellt.
    Diese Normale ist nach au\ss en gerichtet, weil
    \begin{equation*}
        \forall x \in \partial \Omega \; \forall s > 0 \colon
            x + s \cdot 2x \notin \Omega
    \end{equation*}
    Weil alle Punkte~\( x \) die Norm~\( 1 \) haben,
    ist~\( \nabla g \) zu normieren trivial,
    und wir erhalten das nach au\ss en gerichtete Einheitsnormalenvektorfeld
    \begin{equation*}
        \NormalFunc \paren{x} = x
    \end{equation*}

    Wir k\"onnen~\( \partial \Omega \) parametrisieren durch
    \begin{equation*}
        \Phi \colon \IntervalCO{0}{2 \pi} \times \IntervalClosed{0}{\pi}
            \to \RealNums^3
        , \enspace
        \begin{pmatrix} \varphi \\ \theta \end{pmatrix}
            \mapsto \begin{pmatrix}
                \sin \paren{\theta} \cos \paren{\varphi} \\
                \sin \paren{\theta} \sin \paren{\varphi} \\
                \cos \paren{\theta}
            \end{pmatrix}
    \end{equation*}
    Damit erhalten wir
    \begin{align*}
        S &
        = \int_{\partial \Omega} V \vecdF
        = \int_{\partial \Omega} \InnerProd{V}{\NormalFunc} \dF
        \displaybreak[0] \\ &
        = \int_0^{2 \pi}
            \int_0^\pi
                \InnerProd[\Bigg]{
                    \begin{pmatrix}
                        \sin^2 \paren{\theta} \cos^2 \paren{\varphi} \\
                        \sin^2 \paren{\theta}
                            \sin \paren{\varphi} \cos \paren{\varphi} \\
                        \sin \paren{\theta} \cos \paren{\theta}
                            \sin \paren{\varphi}
                    \end{pmatrix}
                }{
                    \begin{pmatrix}
                        \sin \paren{\theta} \cos \paren{\varphi} \\
                        \sin \paren{\theta} \sin \paren{\varphi} \\
                        \cos \paren{\theta}
                    \end{pmatrix}
                }
                \norm[\Big]{
                    \frac{\partial \Phi}{\partial \varphi}
                            \paren{\varphi, \theta}
                    \times
                    \frac{\partial \Phi}{\partial \theta}
                            \paren{\varphi, \theta}
                }
            \dx[\theta]
        \dx[\varphi]
        \displaybreak[0] \\ &
        = \int_0^{2 \pi}
            \int_0^\pi
                \paren[\big]{
                    \sin^3 \paren{\theta} \cos \paren{\varphi}
                    + \sin \paren{\theta} \cos^2 \paren{\theta}
                        \sin \paren{\varphi}
                }
                \,
                \abs[\big]{\sin \paren{\theta}}
            \dx[\theta]
        \dx[\varphi]
        \displaybreak[0] \\ &
        = \int_0^{2 \pi} \cos \varphi \dx[\varphi]
            \int_0^\pi \sin^4 \theta \dx[\theta]
        + \int_0^{2 \pi} \sin \varphi \dx[\varphi]
            \int_0^\pi \sin^2 \paren{\theta} \cos^2 \paren{\theta} \dx[\theta]
        = 0
    \end{align*}
\end{document}
