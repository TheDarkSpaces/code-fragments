\documentclass[../full]{subfiles}


\begin{document}
    \MainEx[Komplexes Kurvenintegral]{2}

    \begin{gather*}
        \gamma \colon \IntervalClosed{0}{\pi} \to \ComplexNums
        , \enspace
        t \mapsto \eEuler^{\iComplex t} + \eEuler^{\iComplex t} \cos t
            = \eEuler^{\iComplex t} \paren{1 + \cos t}
        \\
        f \colon \ComplexNums \to \ComplexNums, \enspace z \mapsto z^3
    \end{gather*}

    \SolutionType{Stammfunktion}

    Offensichtlich ist~\( f \) holomorph
    und hat die Stammfunktion~\( F \paren{z} = \frac{z^4}{4} \).
    Damit gilt
    \begin{equation*}
        \int_\gamma f \ds[z]
        = F \paren[\big]{\gamma \paren{\pi}} - F \paren[\big]{\gamma \paren{0}}
        = F \paren{0} - F \paren{2}
        = 0 - \frac{2^4}{4}
        = -\frac{16}{4}
        = -4
    \end{equation*}

    \SolutionType{Umparametrisierung}

    Weil~\( f \) holomorph ist
    mit offensichtlich einfach zusammenh\"angendem Definitionsbereich,
    d\"urfen wir anstelle von~\( \gamma \) eine beliebige Kurve verwenden,
    solange die Endpunkte~\( \gamma \paren{0} = 2 \)
    und~\( \gamma \paren{\pi} = 0 \) \"ubereinstimmen.

    \begin{gather*}
        \widetilde{\gamma} \colon \IntervalClosed{0}{\pi}
        , \enspace
        t \mapsto 2 - \frac{2t}{\pi} = 2 \paren[\Big]{1 - \frac{t}{\pi}}
        , \qquad
        \widetilde{\gamma}' \paren{t} = -\frac{2}{\pi}
        \\
        \int_\gamma f \ds[z]
        = \int_{\widetilde{\gamma}} f \ds[z]
        = \int_0^\pi -\frac{16 \paren{1 - \frac{t}{\pi}}^3}{\pi} \dx[t]
        = \frac{16}{\pi} \int_0^\pi
            \frac{t^3}{\pi^3} - \frac{3 t^2}{\pi^2} + \frac{3t}{\pi} - 1
        \dx[t]
        = \frac{16}{\pi} \FunctionBounds
            {\frac{t^4}{4 \pi^3} - \frac{t^3}{\pi^2} + \frac{3 t^2}{2 \pi} - t}
            {0} {\pi}
        = -4
    \end{gather*}

    \SolutionType{Kurvenintegral}

    Um \"uber~\( \gamma \) zu integrieren, halten wir zun\"achst fest, dass
    \begin{gather*}
        \gamma' \paren{t}
        = \eEuler^{\iComplex t} \paren{\iComplex + \iComplex \cos t - \sin t}
        = \iComplex \eEuler^{\iComplex t} \paren{1 + \eEuler^{\iComplex t}}
        = \iComplex \paren{\eEuler^{\iComplex t} + \eEuler^{2 \iComplex t}}
        \displaybreak[0] \\
        \begin{aligned}
            \cos^2 t &
            = \frac{\eEuler^{2 \iComplex t} + \eEuler^{-2 \iComplex t} + 2}{4}
            = \frac{1 + \cos 2t}{2}
            \\
            \cos^3 t &
            = \frac{
                \eEuler^{3 \iComplex t} + \eEuler^{-3 \iComplex t}
                + 3 \eEuler^{\iComplex t} + 3 \eEuler^{-\iComplex t}
            }{8}
            = \frac{\cos 3t + 3 \cos t}{4}
        \end{aligned}
        \displaybreak[0] \\
        \begin{aligned}
            \paren{1 + \cos t}^3 &
            = \cos^3 t + 3 \cos^2 t + 3 \cos t + 1
            = \frac{\cos 3t + 3 \cos t}{4}
                + \frac{3 + 3 \cos 2t}{2} + 3 \cos t + 1
            \\ &
            = \frac{\cos 3t + 15 \cos t}{4} + \frac{3 \cos 2t + 5}{2}
        \end{aligned}
    \end{gather*}
    Des Weiteren benutzen wir ein paar Integrale:
    \begin{align*}
        \int \eEuler^{ct} \sin t \dx[t] &
        = -\eEuler^{ct} \cos t + \int c \eEuler^{ct} \cos t \dx[t]
        = c \int \eEuler^{ct} \cos t \dx[t] - \eEuler^{ct} \cos t
        \displaybreak[0] \\
        \int \eEuler^{ct} \cos t \dx[t] &
        = \eEuler^{ct} \sin t - \int c \eEuler^{ct} \sin t \dx[t]
        = \eEuler^{ct} \sin t - c \int \eEuler^{ct} \sin t \dx[t]
        \\ &
        = \eEuler^{ct} \sin t + c \eEuler^{ct} \cos t
            - c^2 \int \eEuler^{ct} \cos t \dx[t]
        = \frac{\eEuler^{ct} \paren{\sin t + c \cos t}}{1 + c^2}
        \displaybreak[0] \\
        \int \eEuler^{c_1 t} \cos c_2 t \dx[t] &
        = \frac{1}{c_2} \WithCondition*{
            \int \eEuler^{cu} \cos u \dx[u]
        }{\begin{subarray}{l} u = c_2 t \\ c = \frac{c_1}{c_2} \end{subarray}}
        = \frac{1}{c_2} \WithCondition*{
            \frac{\eEuler^{cu} \paren{\sin u + c \cos u}}{1 + c^2}
        }{\begin{subarray}{l} u = c_2 t \\ c = \frac{c_1}{c_2} \end{subarray}}
        \\ &
        = \frac
            {\eEuler^{c_1 t} \paren{\sin c_2 t + \frac{c_1}{c_2} \cos c_2 t}}
            {c_2 + \frac{c_1^2}{c_2}}
        = \frac
            {\eEuler^{c_1 t} \paren{c_2 \sin c_2 t + c_1 \cos c_2 t}}
            {c_1^2 + c_2^2}
    \end{align*}

    Wir bemerken dass
    \begin{equation*}
        \forall n \in \NatNums \colon
        \int_0^\pi
            \paren{\eEuler^{4 \iComplex t} + \eEuler^{5 \iComplex t}} \cos nt
        \dx[t]
        = \frac{4 \iComplex \paren{\cos n \pi - 1}}{n^2 - 16}
            + \frac{5 \iComplex \paren{\cos n \pi + 1}}{25 - n^2}
    \end{equation*}
    und erhalten damit
    \begin{align*}
        \int_\gamma z^3 \dx[z] &
        = \int_0^\pi
            \eEuler^{3 \iComplex t} \paren{1 + \cos t}^3
            \cdot \iComplex
                \paren{\eEuler^{\iComplex t} + \eEuler^{2 \iComplex t}}
        \dx[t]
        = \frac{\iComplex}{4} \int_0^\pi
            \paren{\eEuler^{4 \iComplex t} + \eEuler^{5 \iComplex t}}
            \paren{\cos 3t + 6 \cos 2t + 15 \cos t + 10}
        \dx[t]
        \displaybreak[0] \\ &
        = \frac{\iComplex}{4} \paren[\Bigger]{
            \begin{aligned}
                \mspace{24mu} & \mspace{-24mu}
                \int_0^\pi
                    \paren{\eEuler^{4 \iComplex t} + \eEuler^{5 \iComplex t}}
                    \cos 3t
                \dx[t]
                + 6 \int_0^\pi
                    \paren{\eEuler^{4 \iComplex t} + \eEuler^{5 \iComplex t}}
                    \cos 2t
                \dx[t]
                \\ & {}
                + 15 \int_0^\pi
                    \paren{\eEuler^{4 \iComplex t} + \eEuler^{5 \iComplex t}}
                    \cos t
                \dx[t]
                + 10 \int_0^\pi
                    \eEuler^{4 \iComplex t} + \eEuler^{5 \iComplex t}
                \dx[t]
            \end{aligned}
        }
        = \frac{-\paren{\frac{8}{7} + \frac{60}{21} + 8 + 4}}{4}
        = -4
    \end{align*}
\end{document}
